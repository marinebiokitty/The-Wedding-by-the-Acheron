\documentclass[blue]{Kos}
\begin{document}
\name{\bGeneralHistory{}}

Once, long ago, these were barbarian lands, ruled by nomad tribes and conquered by a steady string of foreign horse-lords and fortress-barons. The Sabine people, such as they were, were little more than landless and ragged wanderers, beholden to the whims of the brutal armies and warlords who ravaged their barren realm. Then, many centuries before our time, the tide turned against the savage invaders. The rulers of an oasis kingdom at the heart of the Sabine chaparral mastered the art of divination magic, enabling them to distinguish between truth and lies, ensuring the loyalty of their vassals and defending them against the treachery of their foes. The wisest of their number could even tap into the currents of the future, revealing prophecies not meant for mortal eyes. With this power in their hands, they unified the scattered tribes of the Sabine heartland, rallying warrior clans and city-states into the seeds of an empire that would endure for generations, based at the city of Lavinium. 

When these founders of the Sabine Empire finally passed into eternity, they handed down a token of their authority: the fabled Diadem of the Sabines, an enchanted artifact that endowed its wearer with immense magical power. The longer a royal of the Sabine bloodline wore the Diadem, the more their power increased - yet if a usurper without true royal blood dared to place the artifact on their brow, it would inflict a terrible curse upon them. Yet even an empire of such legendary power could not endure forever. Bogged down by a ponderous bureaucracy, outbreaks of rebellion, and economic stagnation, the only way for the Sabine states to survive was for the Empire to die. And so, 300 years ago, the decision was made. The Sabine Empire, crippled by its own girth, carved itself into the three smaller nations that survive to this day: Etruria in the west, Assyria in the east, and Scythia in the center.

Yet even the division of the Empire could not resolve every tension that had plagued the Sabine lands. The provinces that became Etruria and Scythia had been bitter rivals even in the empire's heyday, and their relations swiftly devolved from simmering tension to outright hostility. Bloodshed was never far from their minds, and the history of the two nations has been fraught with brutal feuds and petty grudges ever since. The first \cEmeraldQueen{\monarch} of Scythia, a bold warrior and hoarder of emeralds named Smaragdos, vied continually with the new Etruscan king over the fate of the legendary Diadem. Neither of them could maintain control of the contentious relic - a series of heists, battles, and bargains meant that it changed possession almost constantly. Eventually, it disappeared from the Etruscan treasure room for the last time; some say the aging Smaragdos presided over the theft \cEmeraldQueen{\themself}. That was the last anyone saw of the Diadem, or of Smaragdos: \cEmeraldQueen{\their} convoy disappeared at sea under suspicious circumstances, and both the \cEmeraldQueen{\monarch} and \cEmeraldQueen{\their} prize were widely mourned. 

Naturally, the blame for Smaragdos' death fell to Etruria, heightening the aggression between the fledgling nations. Retaliation on both sides was continuous and ruthless, until the rulers of both nations had forgotten what had sparked the feud in the first place. Skirmishes, petty duels, corporate espionage, sabotage, piracy, robbery, and embezzlement ravaged both nations, stifling their development. In the centuries that followed, Etruria was beset by a string of voracious plagues, famines, and earthquakes (some say it was the god’s revenge for the murder of Smaragdos), leaving the nation proud but poor. Though Scythia fared better, it remains only moderately wealthy. Assyria, on the other hand, is a realm of luxury and wisdom: unharried by the internecine disputes of its neighbors, it dedicated itself instead to social and educational advancement, crafting a democratic meritocracy that swiftly grew rich off of foreign trade. The universities and academies of Ashur and Nineveh are renowned for their scholarship and rigor, and young Sabine royals of both families often attend Assyrian schools until they come of age - though \cGroom{} famously chose to remain in Etruria for \cGroom{\their} education instead.

Yet passing between the nations and putting down new roots is a complex and laborious process. Though the customs agencies and border guards of each nation behave differently, the procedure for immigration is almost always incredibly frustrating, requiring a massive fee (to enter Etruria) or the signature of a monarch themselves (to enter Assyria). Scythia, in fact, requires both. The alternative, however, is just as inconvenient. Without proper citizenship, it's nearly impossible to get a job more comfortable than exhausting day labor, and all non-citizens must return to their home country and renew their paperwork once every year. Nor is forging such a document easy - important contracts, including immigration agreements, are themselves magical and cannot be forged or copied: the paper itself is tinted red with arcane dyes, and will only turn green once the true individuals whose names are required have signed.

Scarcely a century ago, Etruria seemed on the brink of collapse: its economy in shambles, its borders harried by foreign invaders, its court plagued by civil strife. The nation managed to survive, but its ruling dynasty, House Serta, paid the price: a distant cousin, Hyperion Coronus, seized power in a bloody coup and put the last remaining Etruscan royals to the sword. Now his \cEtruriaKing{\offspring}, \cEtruriaKing{}, rules the nation - \cEtruriaKing{\their} dicta are brilliantly effective, but \cEtruriaKing{\their} morality is rumored to be nonexistent. Citizens whisper of the dreaded prison camps - horrible dungeons where those who dare to question \cEtruriaKing{\their} reign are sent to rot behind the iron grates. They say you're lucky if the gangrene gets you before the rats do, or if the rats kill you before the torturers. But these are only whispers, and no one knows the truth. Every so often some ragged half-mad beggar will claim that they've escaped from the camps, but their stories are hardly given credence by more rational members of society. Short of an outright coup, however, it is impossible for a king or queen to lose their royal status. Nevertheless, their power is not absolute: their expenditures are carefully controlled by the royal treasuries, and they are required to justify any needless expenses beyond a strict budget. Embezzlement, even royal embezzlement, is strictly punishable by life imprisonment under all circumstances.

One thing is certain, though: bad blood still runs between Scythia and Etruria, and they now stand on the brink of war. It all began with the death of the Etruscan queen, Cerintha, a sweet and caring queen who was much-loved by her people. Her habit of paying surprise visits to impoverished border towns, feeding and clothing all those who dwelled there, made her the idol of every poor Etruscan. But on one such visit, to the village of Hero, her convoy was ambushed by a Scythian phalanx, and she was killed by accident in the chaos that ensued. Etruria demanded vengeance. So when \cFugitive{\Prince} Thoesi of Scythia, the ten-year-old \cFugitive{\offspring} of \cScythiaKing{\Monarch} \cScythiaKing{} and \cScythiaQueen{\Monarch} \cScythiaQueen{}, disappeared from \cFugitive{\their} bed one night and was declared dead by the Soulblades soon afterwards, it was \cEtruriaKing{} who took the blame, though \cEtruriaKing{\they}’s never admitted that \cFugitive{\their} blood stains \cEtruriaKing{\their} hands.

\cScythiaQueen{}, Scythia's \cScythiaQueen{\monarch}, has been full of heartbroken rage ever since \cScythiaQueen{\their} \cFugitive{\offspring}'s death. The tensions between the nations have not been defused since - border clashes have continued to claim hundreds or even thousands of lives, the hot-blooded Etruscan \cPoet{\prince} \cPoet{} knocked out half a Scythian diplomat's teeth with \cPoet{\their} bare hands to great Etruscan cheers, there are hints of an Etruscan spy in \cScythiaKing{\Monarch} \cScythiaKing{}'s court, and no one is willing to back down. This wedding is not a mark of mutual respect or admiration. Instead, it is driven by a fear of divine vengeance.

Less than two years ago, the reclusive Cumaean Oracle emerged from her temple after a pair of devastating hurricanes ravaged the coastline, uttering dire warnings in a prophetic trance. The will of the gods is clear, she said: the old blood feud must be reconciled. An Etruscan noble and a Scythian noble, no more than ten years apart in age, must be bound by matrimony or the wrath of Olympus will tear the nations asunder and the rivers will run red with blood and flame. Six months ago, a decision was made. Dreading the prospect of civil war without end, the monarchs of the two kingdoms desperately agreed to arrange the marriage of their heirs, though whether either will follow through on this promise remains to be seen. The announcement spread like wildfire through the Sabine kingdoms: a Scythian \cBride{\prince} was to marry an Etruscan \cGroom{\prince}.

The divination magic that once ran through the veins of the Sabine emperors still empowers the scions of the three kingdoms. In some, it manifests as a powerful ability to compel others to speak the truth: once you see a king bathed in the light of magic, piercing your soul with their ashen gaze, you cannot resist answering their questions and answering true. In a few others, this power appears as the ability to tell the future, gleaning bits and pieces of the fortunes of others. Although truth-telling magic is passed down by blood alone, it is possible for even a lowly commoner to master fortune-telling magic given years of intense training and a healthy dose of luck. Indeed, the process of learning divination is so time-consuming and agonizing that few royals these days even bother. Some people, though, have a natural talent for fate and empathy, and it is these gifted few who often serve as the fortune-tellers present at royal weddings.

Divination magic is not the only form of secret power to be found in the Sabine kingdoms. In Scythia can be found the ancient order known as the Soulblades, renowned healer-priests who inhabit the Soulblade Quarter of the Scythian capital. These scholars and arcanists dabble in the art of ``Soul Magic,'' which many have condemned as the practice of necromancy. Their most powerful spells, which are said to tap into the souls of the dead, are volatile and dangerous, and even the most controlled ritual can backfire suddenly and send its practitioner to meet Hades before their time. 

But the Soulblades were not previously allowed to indulge in these practices unchecked. They have been opposed for centuries by the mythic Blackguards, fearsome assassins who considered Soulblade magic far too dangerous for these reckless dilettantes to indulge in. They dedicated themselves to stamping out Soul Magic wherever it appeared, using ingenious but entirely nonmagical methods to kill the Soulblades and stifle all knowledge of their magic. The Blackguards protect their secrets jealously, and to discover the identity of a Blackguard is said to be certain death.

After the collapse of the Sabine empire, the Blackguards lost their imperial patronage, and the Soulblades consolidated their power exclusively in Scythia. They rebranded themselves as healers and took root in the Scythian court, where they took vicious pleasure in eliminating any Blackguards who remained. While they now profess to be devotees of harmless healing magic, there are still rumors that some among their number continue to experiment with necromancy, desecrating the remains of the dead in their attempts to speak with - or even resurrect - the spirits of the deceased. Though the Soulblades understandably claim that these rumors are baseless accusations, there are those who still worry. \cScythiaQueen{\Monarch} \cScythiaQueen{}, in particular, is legendary for \cScythiaQueen{\their} hatred of the apparently harmless Scythian Soulblades: in a fit of rage, \cScythiaQueen{\they} had Memnon, one of the most famous Soulblades of modern times, imprisoned and flogged for ``violating the will of nature'' - alleging that he and his compatriots were still delving into necromancy behind \cScythiaQueen{\their} back, and dragging out the macabre evidence to prove it.

\end{document}
