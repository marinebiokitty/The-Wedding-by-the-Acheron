\documentclass[blue]{Kos}
\begin{document}
\name{\bSabine{}}

The three nations of the former Sabine Empire are marked by a curious set of shared rituals, beliefs, customs, and prejudices, some of which date back to before the Empire's schism. One of the most ritualized elements of Sabine culture is the wedding. Scarcely anything takes place at a royal wedding that is not traditional. Ever since the first Sabine kings and queens married off their descendants to neighboring tribes, weddings have occurred exclusively on the first day of summer, the Ides (13th) of June. These marriages are almost always arranged, for the mutual prosperity and peace of the nations involved - although some reckless couples marry for love before their wiser companions can stop them, these unions are considered foolish at best, and self-destructive at worst. 

If two people from different nations are married, they both become citizens of the older partner�s homeland, allowing heirs to cement political allegiances. Ill-advised matches are difficult to undo, however: the concept of divorce is nonexistent, and adultery is strictly forbidden. Scythia, a more conservative nation, also forbids sex before marriage, which Etruscans rarely bother to condemn. The gender of those being married matters not at all - an alliance of two kings or two queens has happened before, and heirs adopted into the line of succession are just as legitimate as trueborn ones, provided they still carry noble blood. Moreover, royals need not marry other royals - they are free to tie themselves to whoever they choose, though most dedicate themselves either to another royal or to a non-noble individual of prodigious renown or talent.

Drinking and gambling are universal: the siblings of the betrothed traditionally organize games of Bluffmaster. This is a bluffing game played at an elegant triangular table, where the assembled nobles can vie against their rivals for pride and profit, fortifying their spirits with alcohol as they play. The wedding cake provided more closely resembles a hearty bread, made of ground spelt harvested from the homelands of both bride and groom. Ambassadors of the common people - interpreted today as the wealthiest merchant of each nation - are invited to attend, and bring lavish gifts for the bride and groom. Fortunetellers, generally commoners with an innate knack for divination who have spent decades mastering their craft, also frequent the festivities, and those who are skilled in the art often learn a great many secrets over the course of the ceremony. A bouquet of roses traditionally graces the venue. 


\end{document}
