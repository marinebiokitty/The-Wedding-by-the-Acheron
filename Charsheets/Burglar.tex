\documentclass[char]{Kos}
\begin{document}
\name{\cBurglar{}}

Born May 20, in the year 272.

    Any book of Etruscan history can tell you how their nation rose to power. Less than a century ago, the Coroni, a tribe of petty and low-ranking nobles from the eastern deserts who still laid claim to royal bloodlines, rose up against the Etruscan lords who had welcomed them in. Led by their would-be king, Hyperion Coronus, they crushed Etruria's peasant militia, sacked its capitol, and put the Etruscan Queen, her retinue, and her family to death. In the years that followed, King Hyperion relentlessly tracked down every trace of the old Etruscan dynasty, seeking to erase their name forever. One by one, the scions of that ancient family were slaughtered, until at last no one was left to avenge the fallen queen - no one who could carry on her lineage, no one who could reclaim her throne. Now, the Coroni rule unchallenged, and the Etruscans that came before them are nothing but a memory. 

    But you believe that all those history books are wrong. And you yourself are living proof. You tell the others that you are \cBurglar{}, the no-name fortune-teller. But you know the truth - you are the last remnant of a royal house that can be traced back to the Sabine Empire itself, in the mythic era of your nation's birth. Those are the stories you tell yourself in your daydreams - that you are the proudest and most regal of the old warrior-queens reborn, that you will single-handedly strike down \cEtruriaKing{} the usurper-\cEtruriaKing{\monarch} and \cEtruriaKing{\their} bloody-handed dynasty and restore Etruria's honor. Do you know any of this for sure? Of course not. But hope is headier and more intoxicating than the strongest drink.

    But other things, sadder things, you know are true. Your mother died stinking of filth and despair, wracked by the pangs of childbirth and dishonored by the brutal squalor of \cEtruriaKing{\Monarch} \cEtruriaKing{}'s prison camps. \cEtruriaKing{\Their} soldiers, unaware of your true identity, cast you out on the rugged slopes of Mount Morophon as food for the vultures and the jackals, hoping that the elements would finish what the Coroni started. As far as \cEtruriaKing{\Monarch} \cEtruriaKing{} was concerned, that was the end of the line: the Etruscan queens were dead and gone forever. But you didn't die. 

    A gang of roustabouts, grifters, and ne'er-do-wells, who had made camp on the mountain's slopes after being cast out of Etruria's capital city, proved themselves that day kinder and more noble than any in the Coroni's royal court. They plucked you from among the jagged cairns and raised you as one of their own. As your adoptive father Cynesios used to fondly say, ``We're all outcasts here, child. The least we could do was take care of another like us.'' As you grew into a young \cBurglar{\human}, his wife, the kind but sharp-witted swindler Neumia, always remarked on how much you resembled Etruria's half-forgotten Queen. 

    Your adoptive family wandered from place to place, camping in sparse forests and dewy vales, and you grew older with the seasons. Unwelcome in cities and shunned by even the poorest farmers, they made a living however they could - performing, begging, stealing. An old fortuneteller, a woman who always claimed to have royal blood, taught you the basics of divination, and you took to it with a startling aptitude - you mastered the art of seeing truth in the chaotic fates of those around you in less than a year. You also took to the art of thievery as soon as you tried it, and became quite skilled indeed at sneaking into the homes and bank-vaults of the rich and depriving them of their dusty treasures. You loved the thrill of the profession, the way your pulse pounded as you crept through empty halls or slipped between the bars of a high window. Even after Cynesios died peacefully in his sleep one moonlit November night, and the troupe drifted apart in sorrow, you stuck to your chosen profession. 

    And you made a pretty penny for yourself along the way - not to mention a reputation for competence and stealth across the Sabine states. You were one of the best - perhaps the best. As time passed, you found yourself highly sought-after by wealthy clients looking to 'liberate' a rival's prized trophy or ancient heirloom. You seldom failed to fulfill a contract - and, though you're wanted for dozens of high-profile burglaries in Scythia, Etruria, and Assyria, the law has only been able to touch you once. And that wasn't your fault. You were infiltrating an Assyrian museum of art with a team of local sneak thieves, only to realize that they'd been paid off to set you up and turn you in to the Assyrian authorities. The death penalty awaited you, if you could not escape. You always knew there was a reason you preferred working alone.

    Thankfully, it seems you have as many friends as enemies. As you cooled your heels in a prison cell (day seventy-three of your imprisonment, as you recall), one of the guards slipped you a scroll alongside your daily gruel. The anonymous letter advised you that a job awaited you once you escaped: it recommended that you insinuate yourself into the retinue of Scythia's \cScythiaKing{\monarch} as a diviner, and attend the royal wedding. Whoever your patron was, they'd done their homework - you hadn't practiced divination in years, except when you used it to plan the occasional job on the side. You'd always had a knack for it, though. Oh, and it mentioned that the guard who'd slipped you the scroll had been bribed to assist in your escape. So escape you did - and as you made your way towards Scythia, more of the unsigned letters followed, filling you in on the details of the robbery. 

Your instructions were simple. One of the wealthiest people attending the wedding is an Etruscan merchant of luxuries named \cMerchant{}. \cMerchant{\They}'s bringing many of \cMerchant{\their} most expensive wares with \cMerchant{\them}, including the fabled Diadem of Smaragdos, lost for generations. So, naturally, it's the Diadem that you're supposed to steal. What a surprise. Your benefactor also provided you with a set of skillfully forged documents establishing you as a Scythian commoner with a talent for fortunetelling, making it child's play for you to join the hangers-on and sycophants of the royal court. Now you're on your way to the wedding. 
    
Once you saw the guest list, you realized that this caper would give you the chance to acquire a vast fortune - enough to return to your ancestral homeland of Etruria. To rob this hapless merchant blind under the noses of royalty will be oh so very satisfying - and what's more, you suspect you know who your employer is. One of the attendees is a Scythian arms dealer named \cArmsDealer{}, a \cArmsDealer{\human} you've heard is an old rival of \cMerchant{}'s and is most likely your patron - and, more importantly, \cArmsDealer{\they} has the power to make you a very rich thief indeed. Your payment for this job has already been tendered - after all, your patron freed you from Assyrian justice, and you owe \cArmsDealer{\them} at least this favor. But maybe you can persuade \cArmsDealer{\them} to shell out a little extra for your assistance - between what you can extract from \cArmsDealer{\them}, and whatever wealth is not tied down (not to mention whatever you can pick up while gambling), you should be able to piece together enough of a nest egg to pay the customs fee of the Etruscan border guards and return home at last.  Favors and offers of eventual payment are no good at all, though - you need to be set to enter Etruria by the time this wedding comes to an end. You've been in the wind, a wanderer with neither nation nor home, for long enough - and now, you have nowhere else to go. You fear Assyrian justice will catch up with you unless you can take refuge as a true citizen of Etruria.

    Finally, the \cEtruriaKing{\monarch} of Etruria, \cEtruriaKing{}, will also be in attendance, as will \cEtruriaKing{\their} two children, \cGroom{} and \cPoet{}. It was \cEtruriaKing{\their} family that imprisoned your mother, brought about her death, and left you to die on a distant mountainside. You have no love for this heartless bastard of a \cEtruriaKing{\human}, and quite a bit of hatred. If you can find any way at all to get back at \cEtruriaKing{} for \cEtruriaKing{\their} father's crimes, you'd very much like to make \cEtruriaKing{\them} pay however you can. But you're not sure how - you're a robber, after all, not an assassin. And revenge, however justifiable, always seems to get people in more trouble than it's worth - so you'd like to focus on the task at hand first. But once you return to Etruria, if the authority of the Coroni has been shaken, perhaps you can slowly reclaim the throne that is your birthright. Vengeance and redemption are at last within your grasp. 

\begin{itemz}[Goals]
\item Steal the Diadem from \cMerchant{} without anyone noticing.
\item Get your hands on enough money to meet Etruria's steep customs fee of 500 coins and re-enter the country.
\item Use whatever means you can to undermine the authority of \cEtruriaKing{\Monarch} \cEtruriaKing{}.
\item Tell the fortunes of the nobles assembled at the wedding until you are able to discern the greater truth hidden beneath the veil of the future.
\item Figure out who hired you to steal \cMerchant{}'s Diadem.
\end{itemz}

\begin{contacts}
\contact{\cEtruriaKing{}} A coldly brilliant monarch, the child of Hyperion Coronus and a usurper of the rightful throne of Etruria.
\end{contacts}


\end{document}
