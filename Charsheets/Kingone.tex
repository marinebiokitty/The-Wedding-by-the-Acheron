\documentclass[char]{Kos}
\begin{document}
\name{\cEtruriaKing{}}

Born December 20, in the year 251.

I was born into a blizzard and therefore named "Hiems"-- "winter"-- and I am as cold as the cruel world I live in. Put in my place, another ruler might have despaired and died off years earlier, but I have survived, dragging my Etruria with me.

I am king of Etruria, the westernmost fragment of the defunct Sabine empire and by far the poorest. Etruscans are a lean and hungry lot, racked by centuries of plague, blight, drought, attacks by barbarians, and-- of course-- ceaseless Scythian harassment, misfortunes that have broken the spirits of many citizens. In previous years, even our rulers have thrown their hands in the air, declaring that our land is cursed by the stars. But my parents, wealthy nobles and distant cousins of those gutless monarchs, held to a fiercer sort of optimism. They cut down the original dynasty and ushered in the era of the Coroni, who rule by pure pragmatism, and Etruria is better for it.

I have followed my parents' example, calculating my every action, never hesitating to do what I deem necessary. Through logic and force of will, I have stood strong against both Scythia and the barbarians of the north. Within Etruria, I have formed an extensive information-gathering network to investigate my country's most prominent figures, many of whom are corrupt. I send most of the offenders to prison camps, but three of the most noxious have had to be more permanently eliminated. Their deaths came secretly, at the hands of the Blackguards.

For centuries, the Etruscan branch of this band of assassins, whose roots trace back to the Sabine empire, has sworn allegiance to only the monarchs of Etruria. The Blackguards are frighteningly effective weapons; they have likely embedded their agents in every major Etruscan institution. I can only guess at the full extent of the Etruscan Blackguards' reach, for I make contact with their leader alone-- a mysterious figure traditionally called the White Rose-- and even our communications are brief and infrequent. We correspond using artifacts from the distant past-- two magical communicators, tuned to respond only to one another. Mine is a beautiful ebony machine, hidden in a secret room in Cerintha�s old quarters. inlaid with an ivory rose and passed down from the admirable Queen Smaragdos, who first compelled the Etruscan Blackguards' obedience. It has 26 keys, each embossed with a letter, allowing me to type in the name of the target and a deadline. I'd guess that White Rose's communicator was more cheaply made and has a simpler interface, with only three buttons-- "P" for "Preparing," "C" for "Completed," or "A" for "Aborted"-- as all his messages contain only one of the three letters. The Blackguards have always prided themselves on their efficacy, so the White Rose has only resorted to the third option once during my reign, when the target official died of natural causes before he could be assassinated. Yet, for all the Blackguards' obedience over the centuries, I still fear their power and call on them sparingly, assigning them only five missions in my thirty-year reign. On three of those missions, they were eliminating the corrupt Etruscan officials. For one assignment, though, I sent them to the east, into Scythia . . .

My own wife, Cerintha, died fifteen years back, when Scythians attacked Hero, the border town where she had been visiting. \cScythiaKing{\Monarch} \cScythiaKing{\nickname}, who had just risen to Scythia's throne, claimed her death was an accident, that he had wished merely to sack the town, not to slay Etruria's queen. But I and all of Etruria knew better. And though I didn't love Cerintha-- she was too impulsive and starry-eyed for my taste, and ours was an arranged marriage born from practicality, not sentiment-- the public had doted on her and was clamoring for revenge. My subjects wanted war, but I chose instead to contact the Blackguards. I ordered them to abduct Thoesi, the ten-year-old crown \cFugitive{\prince} of Scythia, and leave her at an Etruscan citadel. At first, I intended to transport her to the capital city and execute her publicly, yet when my soldiers dragged the \cFugitive{\kid} before me, I realized killing \cFugitive{\them} would be unwise. I feared the effects of a child's execution-- it would shatter the so-called limits of decency, rousing bloodlust on both sides, making war more likely than ever. No, I found an even more prudent solution. My most trusted advisors quietly packed \cFugitive{\them} off to a prison camp, and they then erased \cFugitive{\their} records. The Scythian Soulblades divined that the frail child died a few days later, but I can promise that I did not order \cFugitive{\their} death. Now, even divination magic would be hard-pressed to completely prove my guilt. Convenient, no?

The prison camps are lethal, hopeless places, and \cFugitive{\they} could have died for any number of reasons-- disease, cold, starvation, suicide . . . I'd prefer suicide over wasting away there; it would have been kinder to simply kill Thoesi swiftly in the capitol, though \cFugitive{\their} real fate allows me far more leeway with my words. This is why I have chosen to have the Blackguards assassinate their fifth target, rather than simply imprisoning \cPoet{\them}. I care for this target; I do not wish to cause \cPoet{\them} unnecessary pain. After all, \cPoet{\they} is my younger son.

\cPoet{\They} is a poet, adored by the crowds for \cPoet{\their} resemblance to Cerintha and \cPoet{\their} apparent magnanimity. In practice, however, \cPoet{\they} is an idealistic fool-- more interested in spondees than strategy-- and a dangerous one. \cPoet{\They} gambles whenever \cPoet{\they} can, drinks with abandon every time \cPoet{\they}'s gone away to school in Assyria, and is prone to violent outbursts. \cPoet{\They} rails incoherently against both me and\cPoet{\their}his brother, spouting nonsense about passion and literature and true love. And there was also the time that \cPoet{\they} punched a Scythian diplomat . . . Though I care for \cPoet{\them}, I recognize \cPoet{\they} is a loose cannon, too wild to ever rule a land as fragile as Etruria. On the other hand, \cPoet{\they} is too well-liked to disinherit, so I commanded the Blackguards to eliminate \cPoet{\them} before \cPoet{\they} can ever take the throne. In case anyone suspects foul play, I plan to pin the assassination on Scythia. Strangely, the White Rose sent me the message "P" only a few days ago, signifying that the Blackguards are planning to carry out the deed quite soon. Will they strike at the wedding, I wonder? Why would they target him at such a high-profile event? Exactly what game are the Blackguards playing? There are too many mysteries here for my taste.

My \cPoet{\offspring}'s folly has had only one benefit. Some years back, while studying in the Assyrian school-city of Nineveh, \cPoet{\they} seduced a young \cBride{\human}; one of \cPoet{\their} servants discovered their love letters and sent me copies. I would have taken no notice, had the \cBride{\they} not been Thoesi's younger \cBride{\sibling}, the new crown \cBride{\prince} of Scythia. I contacted \cBride{\them}, informed \cBride{\them} that I knew of her indiscretions, and blackmailed \cBride{\them} into spying on Scythian affairs and reporting back to me. I expected \cBride{\them} to do something desperate-- suicide is not an unheard-of occurrence in such affairs-- but \cBride{\they} instead reacted most reasonably and agreed to work for me. Since then, \cBride{\they} has obeyed me faithfully, feeding me not only Scythian secrets but also advice on political and military strategy, remaining cool-headed in countless crises. I am thankful to have the loyalty of such a remarkable \cBride{\human}, for \cBride{\they} has proved \cBride{\themself} quick-thinking, insightful, and utterly shrewd. Against all odds, I have found that a Scythian \cBride{\prince} half my age is my equal. While my older child, \cGroom{\nickname}, will make a fairly good king, I suspect \cBride{\they} would be the finest monarch Etruria could ever hope to have.

It is because of \cBride{\them} that I have approved of this farce of this �peace-keeping� wedding. For the first time in its history, Etruria is now strong enough that we could win a large-scale military conflict with Scythia; we need not resort to marriage to keep this cold war from heating up. In fact, I have had my military leaders draw up a comprehensive plan to swiftly attack Scythia, loot its banks and treasuries, ruin its various garrisons, and then retreat as swiftly. Our plot hinges on hijacking a ship full of downright fascinating, experimental weapons due to arrive at Scythia no earlier than June 13th-- the reason why I picked June 13th for the wedding date. As soon as the marriage secures Scythia's crown \cBride{\prince} for Etruria, the attack can commence, swiftly crippling Scythia and taking its riches for my own, poor Etruria. However, if we face any resistance, Etruria and Scythia will at last be openly at war, embroiled in a long, ugly struggle that may well bleed us both dry. Unfortunately, my dear \cBride{\prince}-spy says that \cBride{\their} parents have heard rumors of my plans and may somehow respond, depending on the events of the wedding. I must instruct \cBride{\them} to obtain specifics and report back to me. Regardless, I must meet with Scythia's monarchs during the wedding and convince them that the rumors are untrue, that Etruria means Scythia no harm. Perhaps I can also use the Assyrian peacemakers for this purpose, though I fear Sirasu-- their uncannily perceptive international relations secretary-- may question my sincerity.

There are other resources that I hope to gain at this wedding. \cMerchant{\nickname}, the Etruscan representative of the commoners, is my nation's wealthiest luxury goods merchant. Under his slippery charm, \cMerchant{\they} hides a clever mind, and, thanks to my intelligence network, I hear \cMerchant{\they}'s put it to good use, digging up an artifact thought lost for centuries-- the Diadem. Obtaining the Diadem would be a great victory for Etruria, because of its historical significance as well as its magical power, and I have already met with our treasury to determine how much we could feasibly pay for it. Unfortunately, our highest offer would likely fall short of a Scythian bid, but I have a second way of persuading \cMerchant{\nickname}-- I have investigated \cMerchant{\them} and learned \cMerchant{\they} is guilty of embezzlement, a crime punishable by life imprisonment. I am legally obliged to turn over such evidence immediately, but I'd rather stay silent and pressure him into discounting the Diadem. Perhaps I could even win it without paying a single coin.

Finally, I shall seek to exploit the knowledge of the fortune-tellers, who may uncover a great many secrets before this wedding is over. Fortunes are often vague and prone to misinterpretation, yet they can also give me insight into mysteries that might otherwise elude me. Indeed, if I had the time or the empathy to learn fortune-telling myself, I rather suspect I would be invincible . . .

But as things stand, I must tread carefully, for I am no doubt the only schemer at this wedding. Scythia may spy on me, just as I have spied on them-- anyone who tries to worm their way into my confidence or into my country as a whole must be watched carefully. If I succeed in my aims, then Etruria will finally have the glory it deserves. If I fail, I may consider canceling the attack and settling for the status quo . . . But I do not fail.

Goals:
- Attack Scythia and catch them off-guard
- Make sure \cBride{\nickname} becomes the next ruler of Etruria by marrying \cGroom{\nickname}
- Obtain the Diadem for as low a price as possible
- Learn the identities of the Blackguards
- Gather information through \cBride{\nickname} and the fortune-tellers
- Prevent \cPoet{\nickname} from doing anything foolish before dying
- Thwart any Scythian attempt to spy on Etruria

\end{document}
