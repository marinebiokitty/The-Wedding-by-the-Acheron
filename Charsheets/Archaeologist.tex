\documentclass[char]{Kos}
\begin{document}
\name{\cArchaeologist{}}

Born January 10, in the year 281, or so my somewhat dubious birth certificate says.

I was an orphan, passed around by different government institutions, but my unfortunate reality never stopped me from being happy. Since I was young, I've dreamed of strolling through the old Palace of Lavinium and holding the majestic Diadem. I've pretended to stare down my friends and family and know their intentions immediately, through quintessentially royal divination magic. I've wished to attend a royal wedding, marveling at coins pressed with the contracts of kings and queens, consorting with mystical fortune tellers, approaching the altar that shall burn once and only once with the regal, sacred flame \ldots{}

Have I mentioned I'm a fan of the Sabine royal family?

Sadly, I'm from Assyria, which has long since disassembled its monarchy. And while I appreciate the education-fueled meritocracy we've set up in its place, I still fantasize about the golden age of the Sabine Empire, along with its unforgettably glorious rulers. It was that passion for royal history that drove me to become an archaeologist, dedicated to the discovery and preservation of magnificent artifacts from Sabine times. I've built up quite a reputation for myself, too. Despite my youth, I've already participated in five groundbreaking digs-- pun not intended-- and personally discovered a cache of letters between military officers, a long-lost book of verses by great poetesses, and a secret treasure room underneath an ancient temple. These findings have established me as a brilliant archaeologist, and all the schools of Assyria have fought to have me as a student. I've studied mainly in the school-city Ashur, with brief excursions to its neighbor, Nineveh.

Even as I've worked and studied, the Sabine royal family has never been far from my mind. A few years ago, as I wandered the streets of Nineveh, I came across a treasure finer than any relic I've ever found-- \cBride{\intro}, crown \cBride{\prince} of Scythia, directly descended from the ancient rulers. And \cBride{\they} was gorgeous, elegant, lovelier than any legends could convey. But I couldn't even bring myself to approach \cBride{\them}, because I was too stunned to move!

Well, I wilted with embarrassment, but my hopes of meeting the royals weren't crushed forever. Just six or so months ago, the news of the royal engagement struck Assyria-- \cBride{} and \cGroom{}, the first in line to the throne of Etruria and Scythia, respectively, will be married in order to thwart of prophecy of war and assure peace. They'll be bound together by love for each other and for the greater good. Now, what could be more splendid?

The best part of all, of course, was when the details of the agreement were released, and we realized that, in addition to Assyria's utterly brilliant international relations secretary Sirasu, two high-ranking Assyrian scholars were asked to attend as ``peacemakers''-- neutral third-parties who could resolve the disputes that inevitably arise where Etruscans and Scythians meet. Immediately, I was determined to get one of those positions.

I fought tooth and nail, assuaging academic egos, even switching schools and taking on an onerous job as a teaching assistant. Yes, I even moved for this, because the administration at Nineveh was so happy to steal away one of Ashur's star students that they threw their full support behind my candidacy for the position of ``peacemaker.'' Scholarly shenanigans aside, I was quite a reasonable choice. I'm young, like about half of the guests. I'm enthusiastic, as you can probably tell. And, as Assyrians go, I'm quite well-versed in Scythian and Etrurian affairs. Eventually, I won the position, and I've never been so excited! The other peacemaker is some odd astronomy professor named \cAnarchist{}; \cAnarchist{\they} seems rather useless and unmotivated, and I can't help but wonder how \cAnarchist{\they} gained this position. But I won't let some old curmudgeon dampen my zeal.

So it falls to Sirasu and me to keep the peace. This might be difficult, since the Scythian and Etruscan royal families have nothing if not a tense history. Scythia and Etruria have always been at odds, ever since the days of the Diadem. There was a flare-up several years ago, when Queen Cerintha, the one and only spouse of \cEtruriaKing{}, was killed during a battle at Hero, a hotly-contested border town then in the possession of Etruria. Soon afterwards, \cFugitive{}, the ten-year-old crown \cFugitive{\prince}, was kidnapped, in what was widely understood to be Etruria's retribution. The Scythian palace declared that \cFugitive{} had passed away; the Soulblades, powerful sorcerers who now serve the Scythians, had used Soul Magic to ascertain that \cFugitive{\they} was dead. Those few weeks altered both families forever. \cScythiaQueen{}, \cFugitive{}'s \cScythiaQueen{\parent}, and \cPoet{}, Cerintha's somewhat erratic younger child, have seemed to take the loss hardest, while the famously cold-hearted \cEtruriaKing{}, now Etruria's sole monarch, weathered the tragedy with \cEtruriaKing{\their} trademark endurance. Lingering grief over the whole affair will likely cause at least some disturbances during the wedding, but I'll try to de-escalate any conflict. Even if I fail, Sirasu, famed for his canny diplomatic ability, will no doubt step in and keep the peace.

Then there's the issue of the war mentioned in the prophecy. Hopefully, the wedding will prevent violence, but the fact stands that, under \cEtruriaKing{}'s guidance, Etruria is for the first time strong enough to possibly beat Scythia in a military conflict. I wouldn't be surprised if \cEtruriaKing{} is planning to attack Scythia. While I likely shouldn't ask \cEtruriaKing{\them} outright whether \cEtruriaKing{\they} intends military action, I should still communicate with other Etruscans to determine whether there's any real risk of war. Alternatively, the Scythians may know of \cEtruriaKing{}'s plans through their non-negligible intelligence corps, so they could also provide valuable insight. But I'll have to be careful even with the Scythians, for they've had some trouble with information leaks recently, and I wouldn't want them to get the wrong idea about me. I suspect my greatest ally in peacemaking will be \cGroom{\Prince} \cGroom{}, who's known as something of a pacifist, quite unlike \cGroom{\their} \cEtruriaKing{\parent}. \cBride{}'s political acumen could also make \cBride{\them} a helpful comrade, but I don't yet know \cBride{\their} position on war.

My official mandate is diplomatic, but I, of course, have plenty of other plans for this event. After so many years of dreaming, I'm going to a royal wedding! While I've fought off the impulse to start collecting autographs, I still intend to enjoy myself to the fullest. There'll be a clandestine Bluffmaster party, of course, with higher stakes than any card game back in Assyria. I've scraped quite a bit of disposable income together from tutoring less advanced students, so I can play my cards while sipping a cool drink and mingling with princes and princesses.

While I'd of course like to win at Bluffmaster-- if I beat the royals, that'd be a story for the ages!-- I've got an even greater game in mind. Yes, it's the wedding of \cBride{} and \cGroom{}\ldots But why can't I do some matchmaking for myself? There'll be two young, single royals at the wedding-- \cPoet{} of Etruria and \cWard{} of Scythia, and technically \cEtruriaKing{} is also unattached. Now, I can't imagine any greater bliss than marrying one of them and being officially inducted into the world of the royals. Thus, I want to woo one of them-- or maybe all of them! Perhaps, as \cBride{} approaches the altar, walking with \cGroom{}, I'll also stride forward, walking with whoever I can get . . .

I'm intelligent, good-looking, and reasonably likeable, but I admit arranging my wedding within the span of another person's wedding is a long shot. On the bright side, both \cPoet{} and \cWard{} are studying in Nineveh, so I suppose I could settle for just getting a date with them. But I'm really aiming for a guaranteed entrance onto the royal stage, and I know just how to obtain it. Marriages of convenience are a long-established tradition' my background as an archaeologist will provide all the leverage I need.

You see, Queen Smaragdos, daughter of Mel, was the first monarch of Scythia, a wondrous lady renowned for her ambition, daring, prodigious magical skill, and obsession with emeralds. She's perhaps my favorite royal of all, and I've done quite a bit of research on her. I uncovered a dusty tome in a minor library in Ashur, a history book of the lowest sort, filled with bizarre theories and rumors. Most of the contents were mere hearsay, but one bit stands out in my mind-- it claimed that Smaragdos had visited Cos in the days before her death at sea, and that she had brought along the Diadem, the item that originally sparked the Scythia-Etruria feud, in order to hide it on the island where no Etruscan would ever get it. It sounded like something the great queen would do; her competitiveness was unparalleled. Furthermore, a professor of mine who visited Cos a few decades back reported seeing golden sigils, which Smaragdos frequently used as part of her magic, so it seems increasingly possible that she's concealed the Diadem on the island with her magic. I don't dare believe the last part of that old book's claim-- that Smaragdos herself was still somehow preserved on the island-- but it's thrilling to even think that I might be near the Diadem.

Of course, any good archaeologist near the Diadem would have to investigate further, and I will certainly be doing so. For one thing, it'd be the find of the century, and my fame as an archaeologist will be eternal. Secondly-- and maybe more importantly-- it makes for great leverage in my matchmaking! Marriages of convenience are the norm among the royals, and what could be more convenient than marrying a woman who will grant her spouse the Diadem? Thirdly, I could use the Diadem for my peacemaking duties, perhaps convincing Scythia and Etruria to share it as a symbol of their burgeoning friendship. But I don't really trust them to share well, and this last option's too boring for my taste anyway . . .

So I'm going to step into the world of royals, with all its drama, magic, and fantasy. I will keep the peace between two great countries, and I will also retrace Queen Smaragdos' last steps and perhaps recover one of the most fabulous relics of history. Most importantly, I shall play cards and party with all the nobles, and I'll cozy up to one of them-- or maybe more. Here I go!

\begin{itemz}[Goals]
\item Find the Diadem, lost by Queen Smaragdos
\item Marry a royal (the higher-ranking, the better)
\item Help Sirasu keep the peace
\item Hobnob with the royals, preferably beating them at Bluffmaster
\end{itemz}

\begin{contacts}
\contact{Sirasu} The brilliant Assyrian secretary of international relations. You look forward to working with him!
\contact{\cAnarchist{}} A boring old astronomy professor and your fellow peacekeeper.
\end{contacts}

\end{document}
