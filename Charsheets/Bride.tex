\documentclass[char]{Kos}
\begin{document}
\name{\cBride{}}

Born March 3, in the year 279.

I feel like a shadow of a person, constructed from lies. Admirers swarm around me, whispering that my beauty is pure as an angel's, yet my soul is ash. My parents call my agreement to this wedding "self-sacrificing," but it's nothing more than calculated self-preservation. Everyone around me thinks I am carefree, yet I am drowning in panic and pain. Scythia thinks I am their loyal, guileless crown \cBride{\prince}, yet I am Etruria's most valuable spy.

I doubt any diviner could have foretold my predicament. When I was six, Thoesi, my only blood sibling and Scythia's ten-year-old crown \cFugitive{\prince}, was abducted and murdered shortly thereafter. Everyone with any sense knew the Etruscans were responsible, for they wanted revenge for the recent death of their own Queen Cerintha, killed by the machinations of the Elder Gods when Scythians attacked the border town Hero. The Etruscans have caused countless other Scythian catastrophes in the last three centuries, so many around me considered Etruria the root of all their problems. My mother, most of all, thought Etruria was evil incarnate-- I can still see the loathing in her eyes, every time she mentions their land. By all rights, I should have hated Etruria too, but, after a few crying fits as a child, I could never muster up such heated feelings. After all, Scythia has dealt Etruria many blows as well . . .

Regardless, Thoesi is dead, another casualty of the Scythia and Etruria's vicious circle of violence, and, as the only remaining child of the current monarchs, I became first in line to the Scythian throne. Though most future rulers study the arts and sciences-- which make for a cultured monarch, but not necessarily a competent one-- I have prepared diligently for my future career, exploring economics, rhetoric, political science, warfare, and many other relevant topics. Seeing my zeal, my parents let me into official meetings even as an adolescent, allowing me to speak up and offer my own thoughts on crucial issues. When I was eighteen, I went off to Nineveh, Assyria's greatest school-city, to study law for a year. And there, in Nineveh, I met the \cPoet{\human} who would change everything-- \cPoet{\name}, the younger child of Etruria's \cKingTwo{\monarch}. \cPoet{\They} was a poet, a spirited romantic with exquisite dreams of art and true love, of beauty and hope and peace. I didn't care that \cPoet{\they} was \cPoet{\prince} of a land I was supposed to hate. No, I lingered in a fantasy as \cPoet{\they} strolled beside me through Nineveh's streets and sang me poems and promised to love me always, and I vowed I would love \cPoet{\them} always, as well. We were the star-crossed lovers of a grand tragedy, \cPoet{\they} joked. We spent our days together, and the nights . . .

Indeed, misfortune struck, though not the sort we would have expected. When I finished my studies just weeks later and returned to Scythia, I found a letter from \cKingOne{\Monarch} \cKingOne{\nickname} of Etruria awaiting me. A letter that phrased my relationship with \cPoet{\nickname} in bare, sordid terms-- as an affair that would leave my reputation in tatters, for Scythians reject anyone who agrees to sex outside wedlock, for any reason. Chances are that \cPoet{\nickname} betrayed me by accident-- for all \cPoet{\their} brilliance, \cPoet{\they} could be foolish and loose-lipped from time to time-- but the \cKingOne{\monarch} never explained how \cKingOne{\they} knew. Instead, \cKingOne{\They} quite simply stated that \cKingOne{\they} could ruin me by releasing this information. He gave me two options-- 1.) have my indiscretions disclosed to the public, or 2.) feed him secrets from within the Scythian palace. I chose the latter option out of self-preservation. Over the past few years, I have granted him access to highly classified information about economics, politics, court intrigue, and the military. He has asked for my own opinions, on certain occasions, and I have advised him to the best of my ability. As a result, my affair has remained safely secret. 
However, Option 2 comes with its own perils, for anyone who trades away such sensitive risks being caught. Hiems has actually protected me to some extent, evading Scythian inquiries about how \cKingOne{\they} gets \cKingOne{\their} information. Nonetheless, Scythia murmurs about spies in the highest echelons of its government. And every time I hear the rumors, I feel a breathless panic in my lungs; fear of being discovered has goaded me through many insomniac nights. Whenever I do manage to snatch some sleep, I dream of teetering on a cliff�s edge, and someday I will surely fall . . .
I fell into alcoholism for several months, drinking to mask the horror. Stumbling through a groggy haze, I floundered in every sort of shame, certain that I was worthless, that nobody who knew my true self could possibly love me. But one day I snapped out of it, because self-pity wasn�t helping. Only apathy and cool, rational thinking can rescue me now.
	I�m not doomed. I�m an excellent spy and an even better politician, and my value to Etruria may yet extricate me from this mess. \cKingOne{\nickname} has arranged my marriage to his older child, \cGroom{\nickname}, because he wants to steal my political acumen for the Etruscan royal family. He himself has ruled Etruria with calculated pragmatism, shepherding it through its countless crises, though everyone I know denounces him as ruthless and cruel. I alone recognize his actions are necessary, given Etruria�s troubles, and I admire his brilliance. He can rely on neither of his own offspring to continue his legacy-- his older child is too guileless, and the younger too impulsive and emotional, as I know well-- and so he wants me as his successor. I indeed believe that I can rule well, and marrying into the Etruscan royal family would solve many of my immediate problems, for I could stop spying on Scythia and at last serve Etruria openly. My lies-- or at least some of them-- would become truth!
	I�ve never met my new fiance, but we have corresponded on-and-off during our engagement. \cGroom{\They} strikes me as open and generous, like the rumors say, but I have one minor concern-- \cGroom{\they} specifically mentioned in two of \cGroom{\their} letters that \cGroom{\they} values truthfulness above all other qualities. Does \cGroom{\they} suspect me of dishonesty? If so, I must charm all \cGroom{\their} doubts away. I will bat my eyes at \cGroom{\them} and flirt ever so subtly-- most young men melt simply at the sight of me.
	Meanwhile, \cPoet{\nickname} has not written at all. 
\cPoet{\They} must be seething over my engagement, but I try not to think of \cPoet{\them} too much. I fear much more than our romance is in danger. My contacts among Scythia�s intelligence corps suspect that Etruria will launch a large-scale invasion into Scythia immediately after my wedding, and they warn that my parents are now considering military action as well. I have not had the chance to obtain more information, but I reported this much to \cKingOne{\nickname}. The suspicions that \cKingOne{\they} wants to attack are no doubt true, though \cKingOne{\they} has neither denied nor confirmed them himself; Etruria would benefit greatly from striking while Scythia is off-guard (and vice versa). However, if a battle should break out while both nations are prepared for conflict, Scythia and Etruria would end up embroiled in a lengthy war that would drain money and lives from both-- the very nightmare that this wedding is officially supposed to prevent. Personally, I don�t want either country to be crippled by violence. Thus, I will probe both my parents and \cKingOne{\nickname} for more details about their military plans, and I must do my best to steer them all towards peace.
\cQueenTwo{\nickname}, my \cQueenTwo{\parent}, will no doubt want to attack Etruria, as revenge for my \cFugitive{\sibling}�s death. \cQueenTwo{\Their} bloodlust will likely be roused further by \cArmsDealer{\name}, Scythia�s top arms dealer. As Scythia�s richest non-royal, \cArmsDealer{\nickname} will attend the wedding as the �representative of the commoners,� and, from the few times I�ve met \cArmsDealer{\them}, I�ve gathered that \cArmsDealer{\they} would attack Etruria in an instant. \cArmsDealer{\They} claims that \cArmsDealer{\their} hatred of Etruria stems from the death of several cousins in a fight at a small border town, but I suspect \cArmsDealer{\them} of more mercenary motives-- as a weapons dealer, \cArmsDealer{\nickname} would profit greatly from conflict. \cKingTwo{\nickname}, my \cKingTwo{\parent}, probably is more hesitant to go to war, while my betrothed, \cGroom{\them}, is said to be a thorough supporter of peace. There�s one last decided pacifist that I know of-- \cMerchant{\name}, an Etruscan luxury goods merchant with as much to gain from peace as \cArmsDealer{\nickname} does from war.
War is a terrible enough threat, yet I worry as much about another, much smaller-scale danger. Just yesterday, the Soulblades of Scythia sent a letter to me and \cGroom{\nickname}, warning that their historical rivals, the Blackguards, may once again be on the prowl. The Soulblades recently intercepted a lengthy Blackguard-style message but were able to decode just four words: �contract,� �two,� �royal� and �engaged.� �Contract� implies that someone has hired the Blackguards for assassination. �Two,� �royal� and �engaged� suggest that \cGroom{\nickname} and I are the targets.
I haven�t yet decided what to do about this letter. For all I know, it may just be a mean-spirited prank! Though the Blackguards were once world-renowned assassins, they have long since faded into fiction; they haven�t been heard of in Scythia since Sabine times. But I cannot dismiss the message from mind, for I already know who the murderer might be.
\cButler{name} is the steward of the Scythian palace. Though born to Soulblade parents, \cButler{they} turned \cButler{their} back on them at a young age, renouncing \cButler{their} magical heritage, eagerly distancing \cButler{themselves} from the Blackguards� enemies. \cButler{They} has since wormed \cButler{their} way up the palace hierarchy with astonishing speed and become our top servant, yet I have always felt odd around \cButler{them}. \cButler{nickname} has watched me, tracking my movements, showing up near my quarters at the oddest of moments, and whenever \cButler{they} has discussed preparations for the wedding I have discerned a frightening false joy in \cButler{their} eyes. \cButler{They} resembles a bird of prey, circling, preparing to swoop in for the kill. Since I will depart for Etruria soon, \cButler{they} may well strike today, at the wedding.
I need more proof before making a formal accusation, since I�d hate to seem paranoid. But, in truth, I am frightened out of my mind-- the two countries to which I am loyal are heading towards war, my steward is probably planning to kill me, I�m about to marry a man whom I�ve never met, and all I really want is a drink!
And I�ll see \cPoet{\nickname} again. After all these years, I�ve realized that we are wrong for each other. \cPoet{\They} thought of me as \cPoet{\their} muse, a model of virtue and beauty and perfection, but I can never match that image. On the other hand, I want to have a great leader for my partner, but \cPoet{\their} intelligence is too pure to be wasted on politicking. Yet I still long for \cPoet{\their} romantic ideals. I can�t help but want true love . . .

\end{document}
