\documentclass[char]{Kos}
\begin{document}
\name{\cScythiaKing{}}

Born July 12, in the year 253.

The legends tell of an ancient Scythian king named Damocles the Just who commanded his smiths to fashion him a sword keener than a lion?s fangs. When it was completed, his advisers feared that he would lash out against his neighbors, fueled by mad dreams of conquest. Instead, he stood before them, unspooled a single silken thread, and wound it about the hilt. The sword would not be used for war, he decreed. Instead, it would hang above his throne until the day he died. Whenever he or his descendants gazed up at the hanging blade, he hoped that they would be reminded of everything that is at stake whenever a ruler acts ? that even a single reckless misstep or careless blunder could bring bloodshed and disaster crashing down upon the kingdom. A ruler should live out every day as if his life, his people, and his domain hang in the balance ? and he should do everything in his power to preserve them.

    You learned this lesson from your grandfather, who learned it from his grandfather, who, the stories say, learned it from King Damocles himself. For too long you thought it was nothing but a foolish children?s story, just more outdated prattle about ?duty? and ?responsibility.? And then your grandfather died, and then your father ? and then it came time for you, not \cScythiaKing{} the boy but \cScythiaKing{} the \cScythiaKing{\monarch}, to take the throne of Scythia yourself. You remember all too well the first moment you looked up and saw that lethal sword, still hanging suspended above you by a single timeworn thread. The gilded marble of your throne suddenly felt cold as death beneath you. And you realized the crushing weight of the power you now wielded.

    You have borne that weight for more than thirty years now, borne it gravely and with pride. And the people of Scythia have always loved you for it. Your citizens trust, admire, respect, and cherish you ? and you have always strived in turn to keep them safe and content. You were married with great aplomb to an old acquaintance, \cScythiaQueen{}, and bore two children with \cScythiaQueen{\them}, your beloved children Thoesi and \cBride{}. You were happy. And, more importantly, you kept the great nation of Scythia stable and peaceful for years.

    Then, one day, something unexpected happened. You found yourself falling in love for the first time in your long life. A minor noblewoman in the royal court, Musa Astea, caught your eye, captivating you with her mischievous smile and her incredible intellect. Musa was never the most beautiful or most proper of \cScythiaQueen{}?s ladies-in-waiting, but she was undeniably the most fascinating. An irrepressible practical joker, she loved to get the better of her stuffy colleagues, engaging in juvenile ? but brilliant ? pranks at their expense. You loved nothing more than hearing the clarion of her laughter echoing through the halls.

    Whenever Assyria sent emissaries to your court, geniuses and academics all, Musa loved to match wits with their most renowned archaeologists and poets, whiling away her spare time in games of Senet. She almost always won. A venerable Assyrian scholar, watching her play, once pulled you aside with words of praise, remarking, ?If she had only been born in Assyria, my lord, she would be one of the greatest scholars our nation has ever seen.?

    You kept yourself from acting on your affections ? you were a king, after all, and a king never imposes himself upon his subjects ? but you always wondered what might have happened if you had expressed your feelings. Then, one cold autumn night, when \cScythiaQueen{} was off in Assyria on a diplomatic mission, there came a shy knock on the door of your bedchamber. It was Musa. You couldn?t keep from laughing, warm and genuine, more than you had in years. Then she joined in, warming the cold marble stones of the palace with her amusement. It was the first night you shared with her, but not the last. You wished sometimes, in your most tender moments, that you could abandon the charade, abandon even the throne, marry Musa, and retire to a distant home. But the kingship ? your duty to Scythia ? weighed on your shoulders more heavily even than love. And so your moments with Musa were limited to those nights, too few and far between, when \cScythiaQueen{} was away and the palace was quiet.

    Then you overheard a Soulblade attendant, a pompous but well-meaning man named Epidotis, mention that one of \cScythiaQueen{}?s attendants was with child. The woman he spoke of was Musa. Epidotis neither knew nor cared who the child?s father was, but you did. If you were revealed as the father, your reputation as a just and moral king would collapse in an instant, and ? you feared ? Scythia?s stability would crumble with it. You knew the choice you had to make. So you saw to it that Musa was married off to another minor noble several years her junior, Polutrupon, and sent off to a remote town on the Etruscan border, the prosperous hamlet of Achillea. \cScythiaQueen{}, an old friend of Musa?s husband, told you when their first child was born, a son named \cWard{}. No one ? perhaps not even Polutrupon ? ever suspected that \cWard{} was your son.

    Your feelings for Musa were soon the least of your worries. The peace between Etruria and Scythia ? a peace that you had worked for years to maintain ? was beginning to crumble. The nations were still far from all-out war, but a series of border skirmishes and assassinations began to plague your citizens. One of the first towns to be destroyed was Achillea. Etruscan raiders burned it to the ground one fateful autumn day, leaving Musa and her husband dead in the wreckage. Then, in defiance of your explicit wishes, \cScythiaQueen{\Monarch} \cScythiaQueen{} ordered Scythia?s own troops to retaliate, in an attack that destroyed the border settlement of Hero and claimed the life of the Etruscan queen, Cerintha. Only weeks later, your eldest \cFugitive{\offspring}, Thoesi, vanished, kidnapped and slain by Etruria?s nefarious agents. So much death, hanging above your head. \cScythiaQueen{}, if anything, seemed even more scarred by these tragedies than you: once a wise and gentle \cScythiaQueen{\human}, \cScythiaQueen{\they} appeared to go half-mad with grief and rage, retiring to \cScythiaQueen{\their} quarters as you grieved alone. It was only when you forgave \cScythiaQueen{\them} for \cScythiaQueen{\their} rash decision that \cScythiaQueen{\they} finally emerged, swearing revenge against Etruria. You could tell that something kind in \cScythiaQueen{\them} had broken and been replaced by cold fury. To see such bloodlust, such vengeance, in the heart of someone once ruled by reason and compassion, has shaken you deeply.

    One good thing did emerge from the rubble. Despite the devastation wrought on Achillea, Musa?s son \cWard{}, little more than four years old, somehow managed to survive. Desperate to see your \cWard{\offspring}, you took advantage of \cScythiaQueen{}?s friendship with Musa?s dead husband and convinced \cScythiaQueen{\them} to adopt \cWard{} as your own child. In point of fact, it was \cScythiaQueen{\their} idea, a shard of kindness born of grief - and you were quick to support \cScythiaQueen{\them}. On other matters, however, you disagreed. After the destruction of Achillea and your \cFugitive{\offspring}?s death, \cScythiaQueen{} wanted war ? wanted to crush the upstart Etruscans at any cost. But you would have none of it ? you never wanted so much blood on your hands. Broken by sorrow, torn between your duty to Scythia and your duty to your \cScythiaQueen{\spouse} and remaining \cBride{\offspring}, you found solace in glass after glass of wine, until strong drink and stupor were the only things that could truly dull your agony. Your \cScythiaQueen{\spouse}, always a stickler for upright and moral behavior, would condemn you for your weakness, so you?ve taken pains to conceal it from \cScythiaQueen{\them}.

    For \cWard{\their} part, \cWard{} proved to be both a blessing and a curse. Like \cWard{\their} mother, \cWard{\they} proved mischievous, prone to elaborate practical jokes. \cWard{\They} never did learn to like \cScythiaQueen{}, either ? he joined the Soulblades to spite \cScythiaQueen{\them}, and could never accept \cScythiaQueen{\them} as his adoptive \cScythiaQueen{\parent}. Openly, you claimed to detest \cWard{\their} talent for chaos ? inwardly, however, \cWard{\they} reminded you all too much of Musa. \cWard{\Their} defiance of your authority was frustrating at best, and heartbreaking at worst. So you distanced yourself from \cWard{\them}, lest \cScythiaQueen{} realize that you had once loved \cWard{\their} mother. When you sent \cWard{} off to school in Assyria, it was a weight lifted from your shoulders.

    Sadly, \cWard{} seemed prone to bad habits of \cWard{\their} own ? \cWard{\they} shared Musa?s capacity for troublemaking, but never betrayed any hint of the sheer intelligence that had made \cWard{\their} mother so beautiful. The royal court of Scythia had seen fit to provide \cWard{them} with a stipend while \cWard{\they} studied in Assyria, to provide for \cWard{\their} basic well-being. A few weeks ago, you decided to take a look at the receipts \cWard{} was sending back ? and what you saw surprised and dismayed you. According to the documents accumulated by the court, one of two things was true. Either \cWard{} was squandering all \cWard{\their} money on strong drink and lavish parties, or the discrepancy in the receipts suggested something more sinister: that \cWard{\they} was embezzling money from the crown for \cWard{\their} own shady and suspect purposes, a blatant violation of Scythian law that you could not condone. You?re not sure which is worse. Seeing your \cWard{\offspring} indicted for criminal behavior would be shameful ? but watching \cWard{\them} mirror your own descent into alcoholism would be heartbreaking. You?ve been a slave to drink for too long - you could not bear to see \cWard{\them} bound by the same shackles. If you can convince \cWard{\them} to give up his studies and return to Assyria, perhaps you can keep \cWard{\them} from sliding too deep into the pit of drink and despair.

    But perhaps you have a chance to make things right. Now, with \cEtruriaKing{\Monarch} \cEtruriaKing{}? offer to seal a peace by marrying \cEtruriaKing{\their} \cGroom{\offspring} \cGroom{} to your remaining \cBride{\offspring} \cBride{}, you?ve been given a chance to make things right, to salvage the decaying relationship between the two nations. You truly want to believe \cEtruriaKing{\they}?s had a change of heart. You?d like nothing more. But no negotiation with Etruria is ever as simple as it seems, and things have only grown more complex once \cEtruriaKing{} ascended to the throne. As cunning as \cEtruriaKing{\they} is cruel, you suspect that \cEtruriaKing{\they}?d like nothing more than to crush Scythia beneath the heel of \cEtruriaKing{\their} boot ? and you cannot allow \cEtruriaKing{\them} to catch you off guard. 

    The Scythian court is already far more vulnerable than you would like. Despite your best efforts to fill it only with trustworthy and reliable advisors, recent events have suggested that an Etruscan spy has managed to infiltrate your retinue: it seems that Etruria?s one step ahead of you at every turn. Every time you contemplate waging war against Etruria with a spy in your midst, your heart becomes ice and your ribcage turns to ash. Unless you can root out the spy and stop their nefarious betrayal, any war with Etruria will cost too many lives, too much defeat, to even be an option. 

    Thankfully, you already have some idea who the culprit might be. Your chief steward, a soft-spoken but reliable man named \cButler{}, has been too quiet and inconspicuous of late, almost as if \cButler{\they} fears you. Perhaps it?s nothing, just a shy young \cButler{\human} awed by \cButler{\their} \cScythiaKing{\monarch} - but perhaps \cButler{\they}?s tied to \cEtruriaKing{}. Another potential suspect is the most recent member of your retinue, a fortuneteller named \cBurglar{} with a suspiciously nondescript background. Or - oh gods, what if it?s \cWard{}? Can you even trust your own \cWard{\offspring}? It would serve you right, for keeping \cWard{\their} parentage a secret? but you must not assume the worst. You are a king, and kings do not act on gut feelings and nervous whims. You have to know for sure who is responsible: as long as the leak remains, the lives of your fellow Scythians are in grave danger.

    If worst comes to worst, the power to launch a pre-emptive strike against \cEtruriaKing{} rests in your hands. You will do everything you possibly can to prevent this outcome ? so much senseless hate, so much bloodshed ? but if \cEtruriaKing{}? machinations prove too dire, then rallying Scythia?s forces before it is too late may well be your only option. To put it bluntly, you suspect that \cEtruriaKing{\they}?s only indulging in this wedding ritual at all to lull you into a false sense of security as \cEtruriaKing{\they} prepares an attack on all fronts. To prepare for this eventuality, your nation?s greatest artificers have crafted an arcane communications device with which you might inform your generals of your decision. If you decline to use the communicator, nothing will change: no armies marshaled, no battle plans drafted, no swords sharpened or armor girded. If, however, relations between the two factions should deteriorate, you have two further options. If tensions escalate, but a chance for reconciliation remains, you have the power to increase Scythia?s border defenses, preparing for the worst (and risking further escalation) without acting aggressively. But if all else fails, one option still remains: the metaphorical sword of war yet hangs above your head, and all you need to do is sever one thread to unleash it.

    You hope against hope that the wedding will not come to such a disastrous end, however. A large part of this result, you know, depends on you ? so long as you can learn the identity of the spy at your court, remain aware of \EtruriaKing{}? attempts to manipulate Scythia, and keep your wits about you, you may well be able to forestall any Etruscan treachery and keep the peace between the kingdoms. But with war looming, the burden of rulership is heavier than ever upon your shoulders, and you can scarcely endure it unaided. Loath as you are to acknowledge it, the fire of a strong drink will ease your fears, if only for a moment. You need it ? gods be cursed, how you wish you didn?t ? to keep yourself from breaking down utterly. No one must discover you in a moment of weakness, however, or the people will know how frail and desperate you have become. They must see the strong and noble king you strive to be, not the mortal man you are. And you have a personal duty as well: make sure your family stays safe, no matter how chaotic this wedding gets. You cannot bear to lose another loved one.

\begin{itemz}[Goals]
\item Prevent war from erupting between Scythia and Etruria if you possibly can.
\item Ensure that the planned marriage between your \cBride{\offspring} and \cGroom{} occurs without a hitch.
\item Make sure everyone in your family remains safe and happy.
\item Figure out just what \cWard{} is doing with \cWard{\their} stipend, and convince \cWard{\them} to return home to Scythia.
\item Root out the spy among your retinue and ensure that no more information leaks to Etruria.
\item Make sure no one finds out about your alcoholism.
\end{itemz}

\begin{contacts}
\contact{\cEtruriaKing{}} A coldly brilliant monarch, the ruler of Etruria and your greatest adversary.
\contact{\cGroom{}} The \cGroom{\human} slated to marry your \cBride{\offspring}.
\contact{\cScythiaQueen{}} Your \cScythiaQueen{\spouse}, a morally upright and heartbroken \cScythiaQueen{\person}.
\contact{\cBride{}} Your oldest living child and heir, slated to marry the Etruscan crown \cGroom{prince} today.
\contact{\cWard{}} Your bastard \cWard{\offspring} by Musa Astea, a disobedient and rebellious child.
\contact{\cArmsDealer{}} An arms dealer and a close friend of your \cScythiaQueen{\spouse}.
\contact{\cButler{}} Your chief steward, a diligent and quiet young \cButler{\human}.
\contact{Sirasu Noon} A master Assyrian diplomat and your best hope for resolving this conflict peaceably.
\end{contacts}


\end{document}
