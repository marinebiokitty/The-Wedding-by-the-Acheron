\documentclass[char]{Kos}
\begin{document}
\name{\cPoet{}}

Born June 28, in the year 279.

They say I look like Cerintha, my mother. I'm told I act like her, too. We both loved music and art and drama. We both thought about the world as we'd like it to be, a storybook land of passion and gallantry and people who hold to great ideals. 

She was queen of Etruria, and she died when I was six. Scythia launched an assault against the town of Hero, where Mother was visiting, and she was caught up in the fighting. The Scythians claimed her death was ''accidental,'' but I heard the whispers of the servants, the murmuring of the royal advisors-- everyone claimed that Scythia had slain her on purpose. 

\cGroom{}, my older \cGroom{\sibling}, ''adjusted'' well-- I saw \cGroom{\them} weep once at the funeral and never again. The \cEtruriaKing{\monarch}-- my \cEtruriaKing{\parent}, I suppose, but I can hardly bring myself to call \cEtruriaKing{\them} that-- never even shed a tear, as if \cEtruriaKing{\they} was happier alone. \cEtruriaKing{\They} retaliated, yes-- Thoesi, the crown \cFugitive{prince} of Scythia mysteriously disappeared in the night soon afterwards-- but the vengeance felt perfunctory. As \cEtruriaKing{\they} and \cGroom{} carried on with their lives, I cried every night.

My grief was interspersed with rage. When my family's apathy became intolerable, I lashed out, calling out both the \cEtruriaKing{\monarch} and \cGroom{} for their coldness. More publicly, I nourished a grudge against Scythia. One time, one of their diplomats came to Etruria and spoke to us in the most insulting terms, ordering us to ''forget'' all of Scythia's past offenses. I charged at him in the middle of a public speech and punched him in the nose. I'll never forget the cheer that went up among the masses . . .

The Etruscan public loves me, largely because of my resemblance to my mother; I offer a warmth that's all too rare in my family. At home, though, I spy disappointment in my \cGroom{\sibling}'s eyes, and the \cEtruriaKing{\monarch} downright hates me. When I was young, I had few ways to escape their stares. My salvation came in the form of a little book I found in a library, an anthology of poems collected from long-dead Sabine authors. Inside were meditations on life, dirges, odes to the sun and the stars and emotion itself. Contemplating those poems, I felt alive for the first time in years, and immediately I began to compose poetry of my own. Within those couplets, I found the happiness, the grand purpose that had so long missing.

And it was the poetry that finally led me to real-life bliss. Though both the \cEtruriaKing{\monarch} and my \cGroom{} thought I was mad and silently scorned my newfound artistry, my \cEtruriaKing{\parent} deigned to let me pursue it in the Assyrian school-city of Nineveh. I thus exchanged the petty intrigue and gray stagnation of Etruria for one of the most vibrant civilizations in human history. Among throngs of fellow scholars and writers, I practiced my art, realizing that, among all poetry, I adore love poems most. Every love poet has had a beloved who inspires them, and in Nineveh I found my own muse in the form of a visiting law student. From the moment \cBride{\they} stepped through the door, I was infatuated.

Then I learned \cBride{\their} name-- \cBride{}. \cBride{\They} was \cBride{}, crown \cBride{\prince} of Scythia. \cBride{\Their} parents had struck down my mother, and my \cEtruriaKing{\parent} \cBride{\their} \cFugitive{\sibling}. Like Pyramus and Thisbe, we had every reason to hate each other, but I loved \cBride{\them} instead, for \cBride{\they} was too lovely to be defined by \cBride{\their} country. Inexplicably, \cBride{\they} loved me back.

\cBride{} looked beautiful, yes, but \cBride{\their} qualities were not merely superficial. \cBride{\They} was quick-witted and kind and generous of spirit, never holding my family's sins against me. And so, our romance transcended the hatred and politics that divided our countries. We two wandered over Nineveh's cobblestone alleys, across the wooden bridges, past temples fragrant with frankincense and myrrh, everything veiled in a dreamlike glow. We slept together, as well-- Scythians wouldn't approve, but why should their backwards morality have stood in the way of our passion? After we made love, I composed verses upon verses of my finest poetry about \cBride{\them}, joyously writing with an angel beside me. \cBride{\They} finished school only few weeks later and had to return to Scythia, but \cBride{\they} vowed to still love me, though we could not communicate any longer for fear of being discovered. I of course still adore \cBride{\them}-- how could I not? I have continued to write poetry about \cBride{\them}, so that I feel like I have not been without \cBride{\them} at all. Thanks to my poetry, I feel \cBride{\they} has remained with me, and my admiration has only grown.

Even with my poetry and my still-continuing studies, I might have had too much time in Nineveh, time to reflect on my mother's death. Fortunately, the mundane pleasures of university life have kept me from my former gloom. With sadly limited assistance from the royal treasury, I have hosted nightly festivities in my apartments, carousing with my peers, playing Bluffmaster through the night. Just last year, I was joined by a comrade in revelry-- \cWard{}, adopted \cWard{\prince} of Scythia. \cWard{} is a free spirit, painstakingly unprincipled and wickedly fond of cards. \cWard{\They} freeloads off of my parties, never hosting a thing of \cWard{\their} own, and, more often than not, \cWard{\they} walks away with all my guests' money when \cWard{\they} plays Bluffmaster. Still, that scoundrel and I have great fun together. \cWard{} likes \cWard{\their} family about as well as I like mine-- which is to say, not at all-- so we've both kept silent about our personal lives. Though we've had no ''deep discussions''-- honestly, I'm not sure whether \cWard{\they}'s got the requisite intellect-- we enjoy one another's company a great deal.

I've been blessedly happy in Assyria, first with \cBride{}, more recently with my parties and studies and \cWard{}. Etruria, however, will forever be linked to misery in my mind. A few months back, when I returned home for my winter vacation, I was greeted with the worst news I've heard since my mother's death. \cBride{} and \cGroom{} are betrothed.

My \cBride{}-- perfection in human form, the model of every virtue, the muse whom I adore with all my heart-- is being married off by the powers-that-be to my emotionally dead older \cGroom{\sibling}. I accept that there is a prophecy, and that Scythian royal must marry an Etruscan royal. But why them? Yes, as the oldest children of the reigning monarchs, they are the obvious choices . . . But why, why did it have to be them? 

I swear I went mad. I didn't rage at my \cEtruriaKing{\parent}, though \cEtruriaKing{\they} likely crafted this travesty of a marriage only to further some nebulous plan of \cScythiaKing{\their} own. No, I went after \cGroom{}, demanding that \cGroom{\they} withdraw from the engagement, stating that \cGroom{\they} could never love \cBride{} or make \cBride{\them} happy. I called their loveless match a travesty. All through this, \cGroom{\they} just stared at me, not even condescending to reply, and I stalked out of the room.

I drank, waiting for the alcohol to numb me, but it only sharpened the pain. So I stumbled through the palace that night, tottering into my mother's old apartments. There, somehow, I found a secret passageway I'd never seen before. It led to a gorgeous magical communication machine decorated with a white rose-- the sign of the Blackguards, an ancient guild of assassins I had read about in a seminar on Sabine political poetry. And I approached and pressed the keys and asked for \cGroom{} to be eliminated before this twisted, horrific wedding . . .

Then I woke up in my own room. I went, searched my mother's apartments the next day and found nothing-- no machine, no passageway. It could all have been a nightmare. And the machine with its white rose has continued to haunt my dreams, even after I returned to school, though the details of the visions have shifted. Now, instead of me ordering the Blackguards to assassinate \cGroom{}, my \cEtruriaKing{\parent} orders up the death of \cEtruriaKing{\their} wife, some fourteen years ago. ''Make it look like the Scythians' fault,'' \cEtruriaKing{\they} orders.

Could it be true? Could the Blackguards still exist, now serving the monarchs of Etruria? I wouldn't put it past the \cEtruriaKing{\monarch}, whose cold-heartedness is renowned far beyond Etruria, to kill off \cEtruriaKing{\their} own family. And given that both \cEtruriaKing{\they} and Scythia's long-reigning monarchs will also be attending the wedding, I know I will soon be face-to-face with whoever caused my mother's death.

Oh, this wedding . . . \cBride{} is no doubt heartbroken, and \cBride{\they} has not written, for what can \cBride{} say? And \cGroom{} and I stopped talking after the night of my outburst, for what do what to say to \cGroom{\them}? That \cGroom{\their} betrothed and I are in love? That I may have marked \cGroom{\them} for death? I am now sailing to Cos from Nineveh, and I hope to see \cGroom{\them} alive and healthy when I disembark. I don't want \cGroom{\them} dead, and certainly not because of me. But even if \cGroom{\they} is unharmed, I cannot let \cGroom{\them} go through with this wedding. I will meet with my love as soon as I can, and we must plot a way to break up this engagement. Ideally, I will simply replace my \cGroom{\sibling}, and our bond of love will be sanctified by marriage! I won't let my \cEtruriaKing{\parent} take \cBride{} from me, just as \cEtruriaKing{\they} may have taken my mother. 

Try as I might, I can't stop thinking about my mother's death. What really happened at Hero? Who was responsible for stealing her away from Etruria and from me? I know that whoever caused her death will be present, and I intend to learn the truth; I have limited divination powers to help me do so. And I manage to find out, then that person will not leave the island-- while I normally shun violence in favor of verses, my wrath is fearful. I will avenge my mother, once and for all.

There's still more happening at this wedding, since \cWard{} and I-- as the siblings of the engaged-- are obliged by youthful tradition to run the real ''wedding festivities,'' involving plenty of alcohol and high-stakes Bluffmaster. I'll be bringing in money for my bets and some of the drinks; \cWard{} promised to buy some drinks \cWard{\themself}, for once. To top everything off, I did some reading on Cos and discovered that, according to a number of ancient poets, this island offers special opportunity to poets, granting them the power to even save the sick through their art. I do not know quite how such a thing could be, but I fully intend to grasp my chance and prove the power of poetry.

Alas, I have no idea what will happen at this wedding. At the very least, I'll have material for that novel I've always longed to write . . . 

\begin{itemz}[Notes]
  \item Your dreams may not be entirely accurate. You should try and get confirmation before acting upon them.
\end{itemz}

\begin{itemz}[Goals]
\item Break up the engagement and marry \cBride{} yourself
\item Take revenge for your mother's death
\item Save the sick through poetry
\item Run the Bluffmaster party with \cWard{}
\end{itemz}


\begin{contacts}
\contact{\cBride{}} Your beloved muse.
\contact{\cEtruriaKing{}} Your cold, detestable \cEtruriaKing{\parent}.
\contact{\cGroom{}} Your older \cGroom{\sibling}. You are not currently on good terms.
\contact{\cWard{}} Your college drinking buddy, who will be helping you run today's Bluffmaster games.
\contact{\cAssassin{}} A loyal servant of the Etruscan palace, chosen to attend the royal family at this wedding.
\end{contacts} 

\end{document}
