\documentclass[char]{Kos}
\begin{document}
\name{\cGroom{}}

Born June 30, in the year 274.

I grew up as crown \cGroom{\prince} in the palace of Etruria-- not the grandest of palaces, but certainly more comfortable than any other dwelling in my poor country. There were servants to dress me, cook for me, teach me . . . I appreciate these privileges, yet I could never help but wish for something better.

My mother, Cerintha, died when I was eleven and my younger \cPoet{\sibling}, \cPoet{}, was six. Soon afterwards, I lost my \cPoet{\sibling} as well-- though we were both alive and safe, \cPoet{\they} seemed to slip away from the world. I mourned for my mother as well, but \cPoet{\they} seemed lost in a drowning grief I couldn't comprehend. My remaining parent-- \cEtruriaKing{}, \cEtruriaKing{\Monarch} of Etruria-- made matters worse. \cEtruriaKing{\They} chided me for even crying at \cEtruriaKing{\their} wife's funeral; \cEtruriaKing{\their} downright despised \cPoet{}'s unending sorrow. 

\cEtruriaKing{\Parent} considers emotion the enemy of logic, and \cEtruriaKing{\they} considers logic the most important quality in a ruler. I've heard the stories of what \cEtruriaKing{\they}'s done for logic-- \cEtruriaKing{\they}'s lied, manipulated, stolen, faked grief, faked love. \cEtruriaKing{\They} dismisses all traditional virtues as useless, prizing only the abilities to evaluate a situation and act efficiently to benefit oneself. I don't deny that \cEtruriaKing{\their} ruthless utilitarianism has helped Etruria, but I detest \cEtruriaKing{\their} proudly amoral brand of leadership. In defiance of \cEtruriaKing{\them} and \cEtruriaKing{\their} cruelty, I value honesty over all other qualities. Once someone loses my trust, they will likely never win it back. When I succeed to the throne, I suspect Etruria will find my straightforwardness a refreshing change.

I believe in integrity-- a controversial position, during my \cEtruriaKing{\parent}'s reign. Even more unusually, I happen to believe in peace, for no child should ever feel the pain of losing a loved one to senseless violence. While conflict may benefit Etruria in certain cases, war is too risky, too deadly, too evil for me to ever condone. As a pacifist, I have never feared more greatly than I do now. Recently, our generals have begun to say Etruria would profit greatly from attacking Scythia in the near future, provided it caught its enemy off-guard. Recently, my \cEtruriaKing{\parent} announced my engagement to the crown \cBride{\prince} of Scythia with great fanfare, in the name of ``preserving peace.'' I worry that \cEtruriaKing{\Parent}'s up to \cEtruriaKing{\their} usual tricks-- namely, doing the opposite of what \cEtruriaKing{\they} claims. I haven't openly accused \cEtruriaKing{\them} of preparing for war, since \cEtruriaKing{\they}'d only deny it. Instead, I have tried to secretly guess what form \cEtruriaKing{\their} plans for war might take. Since \cEtruriaKing{\they} hasn't already struck, \cEtruriaKing{\They} must be waiting for some specific future event(s). Perhaps Scythia will become more vulnerable, sometime in the near future. Alternatively, \cEtruriaKing{\they} wishes to acquire an advantage, some new resource, before going to war, and \cEtruriaKing{\they} has not yet obtained it. Fortunately, many of the most powerful people in Etruria, Scythia and Assyria will be present as this wedding of mine, and chances are that someone knows more of \cEtruriaKing{\their} shadowy plans. I intend to investigate thoroughly, without attracting my \cEtruriaKing{\parent}'s attention, if I can help it. Once I know who the relevant players are, I must find a way to prevent any flare-ups between Scythia and Etruria.

I suspect the fiercest opponent to peace is \cScythiaQueen{\Monarch} \cScythiaQueen{}, who reportedly has never forgiven Etruria for Thoesi's death. \cScythiaQueen{\Their} bloodlust will likely be roused further by \cArmsDealer{}, a parasitic Scythian arms dealer who, unfortunately, will attend the wedding as the ``representative of the Scythian commoners.'' \cArmsDealer{\They} pretends to genuinely hate Etruria, since several of \cArmsDealer{\their} cousins died when Etruscans attacked their small border town; on the other hand, \cArmsDealer{\they} likely has baser motives, as \cArmsDealer{\they} could profit greatly from a conflict. 

Fortunately, there's a few guests who support peace as wholeheartedly. First of all, \cMerchant{}, an Assyrian-turned-Etruscan luxury goods merchant, has as much profit to gain from peace as \cArmsDealer{} does from war. \cMerchant{} could be a valuable ally, though \cMerchant{\they}'s so oily in \cMerchant{\their} demeanor that I still find \cMerchant{\them} somewhat distasteful. The Assyrian delegation, of course, will be a powerful ally in peacekeeping-- especially Sirasu, whose political intellect is known far and wide.

This wedding. This wedding . . . I know I should embrace it, since the prophecy claims it will help preserve peace. Yet I loathe it in reality. I've never understood why everyone around me is so intrigued by sex and romance-- in recent years, \cPoet{} has thrown \cPoet{\them}self headlong into love poetry, and \cPoet{\their} resulting speeches about ``true love'' never cease to confound me. I always thought I would understand upon meeting ``the perfect person.'' Indeed, \cBride{}, my betrothed, is as perfect as a \cBride{\human} can get. \cBride{\They}'s divinely beautiful and smart to boot; we've begun corresponding since our engagement, and \cBride{\their} letters to me are unfailingly courteous and elegant. I've tried to match \cBride{\their} courtesy. 

But that's all we'll ever be-- courteous. Of course, ours is an arranged marriage, so it'd be odd if we were already madly in love, even though \cBride{\they} seems to have every quality one could want. Yet I suspect that I'll never love \cBride{\them} as a \cBride{\spouse}. As a friend, as a ruler, perhaps. Never as a \cBride{\spouse}.

Is there something wrong with me? Is this my \cEtruriaKing{\parent}'s coldness, passed on to a new generation? \cPoet{} is so hot-blooded by comparison-- I've never been prouder of \cPoet{\them} than when a most condescending Scythian diplomat pounced upon Etruria some years ago and \cPoet{\they} hit him in the face during his pompous speech. My \cPoet{\sibling} hates Scythia with a fiery passion, yet even \cPoet{\they} finds my lack of excitement over this wedding disturbing. The day the engagement was decided, \cPoet{\they} stormed up to me, accused me of near-criminal apathy, and raged that my marriage to \cBride{} would clearly be a loveless farce. I was stunned to realize that my lack of romantic feeling was so obvious. But before I could find anything to say, \cPoet{} had run from the room, clearly revolted, and \cPoet{\they} didn't speak to me before returning to school in Assyria soon after. There's been chilly silence between us, ever since.

I have to talk with \cPoet{}. I love \cPoet{\them}, and I don't want to lose \cPoet{\them} just because of this marriage. \cPoet{\They}'ll be coming to Cos directly from Assyria, and I must find a way to meet with \cPoet{\them} discreetly at my wedding. Perhaps I can catch \cPoet{\them} at the hugely entertaining Bluffmaster party that will no doubt occur behind closed doors-- \cPoet{} will no doubt be overseeing the revelries personally. Though \cPoet{\their} temper frightens me sometimes, I hope we can mend our relationship.

Honestly, I'd like to escape this marriage-- all marriage. Can I find some way out? \cPoet{} is my best chance, if I could perhaps compel \cPoet{} to take my place . . . No, \cPoet{\they} would likely hate the idea of an arranged marriage as much as I do. Moreover, such an unconventional change-- switching out fiances at the wedding!-- would probably cause a great deal of shock.

 If \cBride{\they} expects a lovey-dovey romance, I'll probably panic. But I still have hopes we can build a tolerable working relationship, at least during this wedding. \cBride{\They} is said to have a deep understanding of politics, having studied law in Assyria, as well as an innate gentleness, so we may be able to work together for peace. And even if \cBride{\they} does not support peace, I hope \cBride{\they} will tell me so straightforwardly; I've hinted in two of my letters that I value honesty. I'm no fan of this marriage as is, but I cannot possibly submit to it without trust.

I suspect we may make a powerful team. Others seem to expect this as well, for we are already under attack. Yesterday, the Soulblades of Scythia wrote to me, warning that their old adversaries, the Blackguards, have marked us out for destruction. The Soulblades recently intercepted a lengthy Blackguard-style message but were able to decode just four words: ``contract,'' ``two,'' ``royal'' and ``engaged.'' ``Contract'' implies that someone has hired the Blackguards for assassination. ``Two,'' ``royal'' and ``engaged'' suggest that \cBride{\they} and I are their targets. Indeed, there's been a series of strange incidents in my life, recently-- incidents I'd dismissed as accidents, until now. There was a stray arrow that just barely missed my neck when I went hunting, four or five months ago. There was the cook who keeled over, apparently from a heart attack, while handling ingredients for my favorite dessert. And there was the tile that fell from the palace roof, just as I was walking by . . . 

I'd bet anything that someone has been trying to murder me for months. While they seem fairly incompetent, both \cBride{} and I must be on our guard, for the assassin may well strike again at this wedding. I want to know who wants us dead. Someone who desires war between Scythia and Etruria, most likely . . . Thus, the body count may rise before the conflict even begins. Strangely, an assassination might not even be the strangest thing to occur at Cos, for this little island where my wedding will take place is not simply a picturesque venue for over-expensive parties. It has a strange history all its own.

Though most royals study in Assyria as young adults, I elected to stay at home, studying among the relics of Lavinium, the old capital of the Sabine Empire. There, I discovered an old letter, sent by a port official in the very first years of Etruria, a letter I've kept to myself and not shown to anyone else. It claimed that the Diadem, a long-lost magical item prized by both Scythia and Etruria, was loaded into a boat, on which Queen Smaragdos, the first Scythian queen herself, was secretly sailing. According to the epistle, the boat was bound for the small, neutral island of Cos, for Smaragdos wished to hide the Diadem-- the first source of tension between the two countries-- where her Etruscan rivals would never find it. A few days later, she was declared dead.

Thus, the Diadem, the greatest artifact in the history of Scythia and Etruria, may be safe on Cos, ready to re-enter the drama. It was imbued with powerful magic, blessing descendants of the Sabine royal family with good health and charisma, while inflicting a curse on all others. Of course, many people would wish to get hold of the Diadem. It could make for a powerful political tool, enhancing the popularity of monarchs. It could be a stellar addition to some eccentric's collection, fetching fantastic prices in the market. Personally, I appreciate the Diadem for its vast symbolic value. If Scythia and Etruria decided to openly share the Diadem, perhaps exchanging it every year or so, it would certainly reinforce the peace between our lands.

Smaragdos was known as a magician, skilled in many now-forgotten varieties of sorcery, so I suspect she used golden spellcraft to hide the item from her Scythian rivals. Emeralds may also be involved, as she was famously obsessed with that particular jewel. Thus, I should be on the lookout for strange signs of magic. If I can recover the Diadem for both Scythia and Etruria, I vow to make this source of conflict into a beacon of peace. 

So, I've got to reconcile myself with my \cPoet{\sibling}, arrange peace between two countries, dodge an assassin, and dig up the Diadem, long shrouded in secrecy and powerful enchantments. And still, getting married is the most frightening part . . .

\begin{itemz}[Goals]
 \item Make sure Scythia and Etruria remain at peace
 \item Find the Diadem and, if possible, use it to maintain peace
 \item Thwart the Blackguard assassination plot 
 \item Escape marriage without triggering the foretold catastrophe
 \item Reconcile with your \cPoet{\sibling}
\end{itemz}

\begin{contacts}
\contact{\cEtruriaKing{}} Your sole remaining parent, forever shrouded in secrecy.
\contact{\cPoet{}} Your younger \cPoet{\sibling}, whom you love despite \cPoet{\their} headstrong nature.
\contact{\cAssassin{}} A loyal servant of the Etruscan palace, chosen to attend the royal family at this wedding.
\contact{\cBride{}} Your betrothed, whom you know only through your letters.
\end{contacts} 


\end{document}
