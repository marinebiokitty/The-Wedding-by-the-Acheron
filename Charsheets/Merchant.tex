\documentclass[char]{Kos}
\begin{document}
\name{\cMerchant{}}

What is worth more ? gold, or a king's praise? You don't know for sure. But as long as you can have both, why should you care? As far as you're concerned, most royalty is good for exactly thing: making you richer. You come from a family of Assyrian intellectuals, brilliant men and women who could outthink almost anyone in the fledgling democracy. Compared to them, most kings and queens were nothing but inbred dunces. When you first left your childhood home and settled down in Etruria to start a retail business, you knew this all too well. Democracies couldn't care less about luxury goods, and their purse-strings are notoriously tight. But the royal family of Etruria ? well, offer them enough newfangled opulence, and their coffers might as well be yours. After all, inbred dunces are easy to manipulate. That was the theory, anyways. 

As it turned out, Etruria's present king, a cruel and uncompromising tyrant named Hiems, proved to be anything but an inbred dunce. In fact, you grudgingly admitted, he was nothing short of brilliant. And, though you soon struck up a brisk business with Etruria's citizens ? who could, you earnestly believed, use a little bit of luxury in their lives ? you never could persuade Hiems and the royal court to invest in your products. So when you heard the news of a royal wedding, you jumped at the chance to secure some royal patronage. Etruscan, Scythian, it really didn't matter ? once you had the royal treasury flowing into your business, nothing could possibly go wrong.

Your brother Izdubar confounded your plans, though. Twice as brilliant as you ever were on your best days, it always seemed that he solved every problem that got in his way. You tried to tell yourself you weren't jealous, that you could be as smart as he was. If you could just get him on your side, whatever projects you embarked upon would be a rousing success. But he threw a wrench in things. When you headed off to Etruria, he refused to go with you. Try as you might to convince him to go into business with you, he proved as stodgy as he was brilliant, always sticking his nose into musty tombs and worn-out tomes while you watched the money roll in. By the time you left, he'd lined up a professorship at a prestigious Assyrian university, and dedicated himself to research. Waste of a good mind, you always told him.

Your last conversation was not a happy one. He accused you of caring about nothing more important than profit, and insinuated that you were little better than a common peasant. Stung, you put on an arrogant facade and walked away ? but, deep down, you worry that you've lost your brother's respect forever. And then you discovered that he'd be attending the royal wedding. It was your best chance to see him again. Perhaps it can even be something more than that: perhaps it's an opportunity for you to prove yourself to him, to show him that you're worthy of his admiration, to finally \textit{impress} the imperturbable Izdubar.

So, driven by both profit and brotherly pride, you pulled what strings you had, and managed to secure an invitation to the wedding yourself, held on the distant (and politically neutral) isle of Cos. You packed up a valise (or three) full of your most innovative and fascinating wares, including a lavish gift for the happy couple (a cinnamon-flavored distillation of a rare vintage mead imported from Iberia, packaged in filigreed gold). But your most important offering is a unique piece of history, tailored to the desires of all the Sabine royalty. 

Ever since you were a child in Assyria, you'd heard rumours of an ancient artefact known only as the Diadem, a magical token of royalty that was lost when the Sabine empire crumbled. Wouldn't it be a wonderful achievement if you managed to uncover the Diadem and parade your discovery before the assembled royalties? Sadly, the real Diadem was lost to history. So you persuaded your best employees ? a talented crew of artificers ? to make you the next best thing. They acquired an ornate diadem from the correct historical era, then painstakingly enchanted it to give special powers to every royal of Sabine descent who would be attending the wedding, Etruscan and Scythian alike. If your detractors knew, they might call it a "fake," but you prefer to think of it as a work of art in its own right.

Unfortunately, you're not the only merchant attending the festivities. An old rival of yours, a Scythian arms dealer named Fresi, has managed to worm her way into King Cryseon's retinue. You're familiar enough with this serpent of a woman to know that, where she travels, conflict is never far behind. This is, to put it bluntly, bad news. Nobody buys luxury goods in the middle of a war ? austerity measures are the \textit{worst} ? so, if Fresi manages to push the two nations into direct conflict, you're out of luck. What's worse, it seems like they might not even need Fresi's warmongering to push them over the brink. Oh, the two kings are reasonable enough in their own right ? Cryseon of Scythia's a kindly old man, so obsessed with preserving life that he'd never spark a war on his own initiative, and Hiems of course is too cunning to act rashly. But rumour has it that the border clashes between the two nations have been escalating. 

Thankfully, you've put together a contingency plan of sorts. Your sources have managed to locate one of the most talented thieves in any of the Sabine states, a daring rogue who goes by the name of Delia. You reached out to her anonymously, got her admitted to the wedding under the guise of a Scythian fortuneteller, and commissioned her to stage a daring heist. But she's not going to be robbing Fresi ? she's going to be robbing \textit{you}. Little does she know that all of your goods (especially the Diadem) have been magically warded and trapped by your artificers. Anyone who touches them will be knocked unconscious. That's when you'll arrive on the scene, discover the "treacherous thief," and attempt to pin the crime on Fresi ? who, everyone knows, is notoriously underhanded in her own right. With the value of your wares reinforced by the attempted theft, and Fresi hopelessly discredited, you hope to steer the two nations away from war and towards something infinitely better ? mutual profit.


\end{document}
