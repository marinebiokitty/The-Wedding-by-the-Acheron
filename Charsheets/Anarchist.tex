\documentclass[char]{Kos}
\begin{document}
\name{\cAnarchist{}}

Born May 1, in the year 258.

What you would do to feel alive . . .

The feeling of emptiness started when you were young. Your \cMerchant{\sibling}, \cMerchant{}, wasn't born until you were four, and yet \cMerchant{\they} always seemed to livelier than you. As you sat still and gazed off at horizons, \cMerchant{\they} scampered around the markets of Ashur, the Assyrian school-city where you both grew up. You excelled at the rather quiet subjects of physics and philosophy, while \cMerchant{\they} was the best at debate and sports and drama, declaiming at the top of \cMerchant{\their} lungs even into the night. \cMerchant{\They} would come home from school with armfuls of ribbons and trophies-- \cMerchant{\they} asked you to carry them, they were so heavy-- and you had to content yourself with a single astronomy prize every year.

Education meant everything in Assyria, and, while you did fairly well, \cMerchant{} was thriving. One of the greatest students in \cMerchant{\their} year, \cMerchant{\they} was on track to become an advanced researcher in Ashur, or perhaps a well-respected administrator, or perhaps a star professor. \cMerchant{\They} stole your parents' attention-- all of the world's attention, really-- but you forgave \cMerchant{\them}, because you knew \cMerchant{\they} would selflessly channel all those acclaimed talents into the education system. You put up with \cMerchant{\them} for the greater good. All the while, you worked yourself almost to death, studying fiercely, clawing for some successes of your own, but you had lost that rivalry long ago.

Then \cMerchant{\they} announced, one day, that \cMerchant{\they} had no interest in academia. That \cMerchant{\they} couldn't care less about essays and books, because money was all \cMerchant{\they} really valued. Within a few weeks, \cMerchant{\they} had packed up \cMerchant{\their} bags and joined a trading post, and a few years later, \cMerchant{\they} ran off to Etruria to manage ``the international branch'' of \cMerchant{\their} company. What a fraud!

You fumed at \cMerchant{\them}, and you haven't contacted \cMerchant{\them} since \cMerchant{\they} left home. By that point, you were next to dead, weary from years of panicked, jealous studying. Though the panic slowly subsided, you remained a diligent scholar and eventually became an astronomy professor in the school-city of Nineveh. You found calm as you looked at the stars, for your troubles felt miniscule compared to the infinite depth of the night sky.

Infinity's always held a strange attraction for you. Even as a child, you contemplated the infinite enormity of death, and you fantasized about being one of the ancient Soulblades, mages who personally faced the forces of life and death. That was an empty fantasy, since the Soulblades had long since abandoned Assyria. Yet about ten years ago, you were approached by the Soulblades' greatest rivals-- the Blackguards.

The Blackguards, too, dealt with death. But where the Soulblades timidly requested strength from non-human forces, the Blackguards seized power through frighteningly, purely human talent.  Where the Soulblades tinkered with all manner of risky magicks, the Blackguards depended only on long-perfected mundane methods. The Blackguards, fundamentally, were a proud and ancient league of assassins, forging political alliances by offering their murderous skills. You heard their call and joined them, and there was finally some meaning to your pathless life.

Sadly, the Blackguards have fallen on hard times since their heyday in the Sabine era. While the Soulblades consolidated power in Scythia and lived openly as ``harmless healers''. the Blackguards were driven underground. The Scythian Blackguards withered away, unable to survive in the Soulblades' new stronghold. The Etruscan Blackguards were shamefully dominated by Etruria's royal family, reduced to mere servants, forced to commit assassinations only at the monarchs' beck and call. The Assyrian Blackguards were rather isolated in the aftermath of the Empire's fall, and they never regained any serious political influence at all. They still passed down the beliefs and methods of the Blackguards, but they rarely had any cause to use them. Filled with talents they couldn't utilize, cut off from all their old power, the Assyrian Blackguards turned bitter over time.

In between your days as a professor and nights doing research, you trained in some of the Blackguards' methods, becoming a Half-Blackguard, skilled with roughly half of the Blackguard arts of death. But you never had much opportunity to use your skills outside of the practice rooms. The Blackguards took advantage of your professional position on occasion, asking you to pass messages and deliver special packages at conferences and the like, but you were not called on to carry out a hit.

You had resigned yourself to never realizing your full potential-- until you got the news, one year ago. You had been poisoned by something without your knowledge, perhaps by some toxin in the natural environment, perhaps by the venoms used in your Blackguard training. The toxicity had built up in your system to dangerous levels, and you will deteriorate and die within a year. Already, your body is weakened. Your mind, though, is strong as ever as it despairs over your imminent demise.

You can't just die and be forgotten. You refuse to fade away! You initially turned to the arts of the Soulblade-- secretly, of course, because the Blackguards would ostracize you for ever contemplating the use of Soul Magic. You reviewed old Blackguard reports of Soulblade experiments, and the results were fascinating. For one thing, several Soulblades recorded the existence of ghosts, souls of the dead which still dwell in the world of the living, communicate with humans, and even manipulate physical material. The Soulblades claimed to find ghosts all over the Sabine Empire, but they noted that such shades were most common in a few places-- near the River Cocytus, the Stygian Lake, and the Bay of Acheron. Most intriguingly, they reported that it is possible to gain ghostly power oneself. According to the records, it is possible to summon ghosts by ringing a bell, but you've rung and rung throughout Nineveh, with no result. While you haven't quite given up the hope of surviving as a ghost, you've had to start looking at other options. You came across one other remarkable fact in your research; the Soulblades asserted that one of their most sophisticated remedies could stave off some of the effects of poisoning. However, since your case is so severe, you won't be entirely saved from your early death; the magic would merely restore some of your strength in the meantime.

The Soulblade options failed to stave off your death, so you began to think of the Blackguards once more. And six months ago, when the royal engagement was announced, the Assyrian Blackguards instantly masterminded a daring plan. Both Scythia, with their alliance to the Soulblades, and Etruria, presumptuously grinding their Blackguards into submission, have established themselves as enemies to the Blackguards. And you, between your fatal illness and your fairly lofty status in Assyria's educational system, were perfectly positioned to wreak the Blackguards' revenge. What you and your comrades have in mind will shock the whole world, and the Blackguards, of course, claim credit for the chaos. The Blackguard order will take back its power and splendor and punish its oppressors, and your name, specifically, will be repeated for eternity. 

You shall trigger the prophecy and sink the world into war.

The plan is an elegant one. By twisting the arms of the right school administrators and winning some well-timed donations, you obtained a position as one of the three Assyrian ``peacemakers,'' invited to resolve conflicts at this wedding. "Peacemaker," ha-- if you wish, you can do exactly the opposite, working to set off the prophecy by breaking up the engagement. However, that would be somewhat difficult, given how competent your fellow peacemaker Sirasu Noon is. At any rate, you have a more dramatic plan for preventing the wedding, a plan that appeals to you far more than mere politicking.

You see, a crew of Assyrians will visit the island of the wedding venue just before the event itself, in order to clean and decorate. One of those Assyrians, a servant of the Blackguards, shall find a cranny somewhere and hide a powerful new powdered explosive, stolen from a shipment of experimental weapons belonging to \cArmsDealer{}, a Scythian arms dealer. You are no stranger to lockpicks yourself, so you can liberate the explosive-- a bag of white powder that can be passed off as flour or sugar-- and pack it into the wedding altar, set to start burning ten minutes before sundown. Then, when all the monarchs and nobles and businessmen crowd around for the ceremony . . .

Well, you've always wanted to go out with a bang. And you can't imagine any ending more brilliant than a fiery explosion, one that brings down those fools who dared to subjugate the Blackguards. Your fame will be immortal, as the Blackguards resurge once more in the ensuing chaos.

Interestingly, the Etruscan Blackguards will also be doing some business at this wedding. The Assyrian Blackguards received an encrypted magical communication from the head of your Etruscan counterparts, the White Rose: "One of our members has been engaged to commit two assassinations at the wedding, per our wearying contract with the monarchs of Etruria." They hesitated to send more details, since Scythia's Soulblades may very well have been listening in, but you know how to find your fellow Blackguard. Once you have made contact your comrade, you should discover what their assignments are, for it is your duty as a Blackguard to assist them in all such doings.

You are excited to help them and perhaps use your deadly skills at last, but you must also be careful around the other Blackguard. The Assyrian Blackguards have long been radical, out-of-the box thinkers, while the Etruscan Blackguards are known as purists, obsessed with avoiding accidental casualties during their assassinations. As a result, your fellow Blackguard might not approve of the plot with the explosives. Furthermore, you're tempted to re-try Soulblade magic at this wedding. There will be many distinguished Scythians present, and there might well be find a Soulblade among them. Perhaps you can find one and convince them to give you the chronic poisoning remedy, since additional strength will no doubt be useful if you take part in the assassinations. However, the other Blackguard will likely despise such magical dabbling, so you should conduct those activities as stealthily as possible.

You have other magical plans, of course. After all all, Cos is in the Bay of Acheron, so you may finally find a shade to bless you with ghostly immortality; you've brought along a bell, just in case. 

Most of all, you cannot wait to set your explosive trap for the guests. You have no concerns about knocking off most of your victims, though you don't hate them personally. One of the guests, though, is \cMerchant{}. \cMerchant{\They}'s wormed \cMerchant{\their} way in as the 'representative of the Etruscan commoners'-- a fancy phrase to say that \cMerchant{\they}'s the richest non-noble in Etruria. Even after all these years, you still resent \cMerchant{\them} for making your childhood miserable, and you'd love to ruin \cMerchant{\their} reputation before \cMerchant{\they} burns to death. According to rumors, \cMerchant{\they}'s tackily, filthily wealthy and just as phony as ever, so destroying \cMerchant{\their} credibility shouldn't prove too difficult.

On the other hand, there's one guest whom you don't want hurt. \cWard{} is a Scythian royal, the adopted \cWard{\sibling} of the betrothed, but you know \cWard{\them} simply as the quiet, first-year astronomy student who excelled in your third-year seminar last semester. \cWard{\They}'s clearly talented, a scientific prodigy, and \cWard{\they}'s also a careful student with a seemingly genuine passion for knowledge. You'd rather not waste such a love of learning, so you'll look for ways to save \cWard{\them} from the bomb. Perhaps you can convince \cWard{\them} to leave the area during the ceremony, or you can forcibly remove \cWard{\them}. But in the worst case scenario, \cWard{\they}'ll simply perish. And you will go up in flames alongside \cWard{\them}, this empty life of yours justified by its ardent, glorious conclusion . . .

\begin{itemz}[Notes]
  \item Though Blackguard training required you practice Bluffmaster, you never studied it deeply--\cMerchant{} played it constantly as a child, and so you could never enjoy the game much yourself. However, if you wish to identify the Etruscan Blackguard for some reason, you would do well to look among the most talented players of Bluffmaster.
  \item You've stashed the bell you'll need to summon spirits in your trunk. 
\end{itemz}

\begin{itemz}[Goals]
\item Cause chaos, prevent the royal marriage, and trigger the prophecy
\item Find the other Blackguard, and assist them in their assassinations
\item Hurt \cMerchant{}, perhaps by ruining \cMerchant{\their} reputation
\item Learn more about ghosts and gain power in the afterlife
\item Find someone to give you a Soulblade remedy for chronic poisoning
\end{itemz}

\begin{contacts}
\contact{Sirasu Noon (Maitree Banerjee} the Assyrian secretary of international relations. He is irritatingly good at preventing conflict.
\contact{\cArchaeologist{}} a talented archaeology student and your fellow peacekeeper. \cArchaeologist{\They} seems enthusiastic about \cArchaeologist{\their} position, but also rather inexperienced.
\contact{\cMerchant{}} The bane of your existence.
\end{contacts}


\end{document}
