\documentclass[green]{Kos}
\begin{document}
\name{\gWardSoulMagic{}}


To work an act of Soul Magic, you must:

\begin{enumerate} 
\item Have the ability to perform that type of magic.
\item Get the specific ingredients as well as one serving of food-- alcohol does not count. The food and ingredients are consumed by your magic, unless you specifically know otherwise.
\item Chant the relevant incantation three times without being interrupted. You cannot attempt this step if you have been unconscious at any point in the last twenty minutes.
\end{enumerate}

Immediately after chanting, roll a d20. At this point, your ingredients are consumed. If you roll is above the magic's "Risk" level, you automatically succeed and can go get the resulting potion from the GM. Otherwise, you suffer the result that corresponds to your roll:

\begin{itemize}
\item 1, 2: The magic succeeds, but you and everyone within 1 ZOC take one point of psychic damage.
\item 3, 4: The magic succeeds, but you fall unconscious for five minutes.
\item 5, 6: The magic succeeds, but you and everyone within 1 ZOC become unable to speak anything but gibberish for five minutes (though your listening comprehension skills are intact).
\item 7, 8: The magic fails, wasting your ingredients.  You and everyone within 1 ZOC take 1 point of psychic damage and are rendered unable to speak anything but gibberish for five minutes (though your listening comprehension skills are intact).
\item 9, 10: The magic fails, wasting your ingredients. You and everyone within 1 ZOC take 2 points of psychic damage.
\item 11, 12: The magic succeeds, but you and everyone within 1 ZOC fall unconscious for five minutes.
\end{itemize}
Through careful training and sheer intellect, you have the ability to make the following Soulblade potions with relatively low risk to yourself. Unless you learn otherwise, you cannot perform any other types of Soul Magic. Additionally, while you can tell other people what ingredients and incantations are needed for a potion, you cannot transfer the ability to actually craft the potion.
\begin{itemize}

\item Psychic Healing Potion:
\begin{itemize}
\item Incantation- "Talaina sato virumqueca"
\item Specific Ingredients: 2 servings of alcohol
\item Result: \iPsychicHealthRemedy{\MYname}
\item Item Number: \iPsychicHealthRemedy{\MYnumber}
\item Effect: \iPsychicHealthRemedy{\MYtext}
\item Risk- 5
Envelope: B
\item
\end{itemize}
\item Acute Poisoning Remedy:
\begin{itemize}
\item Incantation- "Calden triasidra somza."
\item Specific Ingredients: \iHerbs{}
\item Result: \iAcutePoisonRemedy{\MYname}
\item Item Number: \iAcutePoisonRemedy{\MYnumber}
\item Effect: \iAcutePoisonRemedy{\MYtext}
\item Risk- 8
\item Envelope: D
\end{itemize}

\item Chronic Poisoning Remedy (Rare):
\begin{itemize}
\item Incantation- "Aroiza mi kairilu shierafu."
\item Specific Ingredients: 4 \iEmerald{\MYname}s (not consumed, even on a failure) and a sewing needle (always consumed)
\item Result: \iChronicPoisonRemedy{\MYname}
\item Item Number: \iChronicPoisonRemedy{\MYnumber}
\item Effect: \iChronicPoisonRemedy{\MYtext}
\item Risk- 10
\item Envelope: E
\end{itemize}
\end{itemize}

\end{document}